\documentclass[onecolumn, compsoc,10pt]{IEEEtran}
\let\labelindent\relax
\usepackage{enumitem}
\input{preamble.tex}  
% \def\V{\mathds{V}}
\def\Appendix{Appendix}
%###################################

\newif\iftikzX
\tikzXtrue
\tikzXfalse
%--------------
\def\jobnameX{zero}
%--------------
\newif\ifFIGS
\FIGSfalse 
\FIGStrue
%--------------
\newif\ifdraftQ
\draftQtrue
%\draftQfalse
%--------------
%###################################
\def\TITLE{\enet: Fast Scalable Pandemic Risk Assessment of \infl Strains Circulating In Non-human Hosts}
%\def\TITLE{\LARGE A Biologically Meaningful Sequence Metric\\For Analyzing Evolutionary Changes\\In Novel Pathogens}
%\def\TITLE{Learning  Mutational Patterns At Scale For\\Analysis Of Sequence Divergence\\In Novel Pathogens}
%\def\TITLE{Learning Mutational Patterns at Scale to Analyze Sequence Divergence in Novel Pathogens}

\def\authore{Kevin Wu}
\def\authora{ Jin Li}
\def\authorc{Aaron Esser-Kahn}
\def\authord{Ishanu Chattopadhyay}

\def\addressa{Department of Medicine, University of Chicago, IL, USA}
\def\addressb{Committee on Genetics, Genomics \& Systems Biology, University of Chicago, IL, USA}
\def\addressc{Committee on Quantitative Methods in Social, Behavioral, and Health Sciences, University of Chicago, IL, USA}
\def\addressd{Pritzker School of Molecular Engineering, University of Chicago, Chicago, IL, USA}
\def\addresse{Committee on Immunology, University of Chicago, Chicago, IL, USA}
\newif\ifdraftQ
\draftQtrue
\draftQfalse


%###################################

\title{\LARGE \TITLE}
\author{\sffamily  \fontsize{10}{12}\selectfont   \authore$^{1}$,\authora$^{1}$,  \authorc$^{2,3}$, and \authord$^{1,4,5\bigstar}$\\                                                                
\vspace{10pt}                                                                   

\sffamily  \fontsize{10}{12}\selectfont                                         
$^{1}$\addressa\\   
$^{2}$\addressd\\
$^{3}$\addresse\\
$^{5}$\addressc                                                                 
\vskip 1em                                                                      
$^\bigstar$To whom correspondence should be addressed: e-mail: \texttt{ishanu@uchicago.edu}.}


\def\hcov{SARS-CoV-2\xspace}
\def\RATG13{RaTG13\xspace}
\def\Appendix{Appendix}
\def\qnet{Enet\xspace}
\def\enet{Emergenet\xspace}
\def\erisk{E-risk\xspace}
\def\qdist{E-distance\xspace}
\def\cov{COVID-19\xspace}
\def\infl{Influenza A\xspace}
%\def\infl{IAV\xspace}


\def\E{\mathcal{E}}
\def\dst{x_\star^{t+\delta}}
\def\dsta{x^{t+\delta}}

\usepackage{flushend}
\externaldocument[SI-]{SI}
% \externaldocument[EXT-]{exfig}
\newif\iftikzX
\tikzXtrue
\tikzXfalse
\def\Extended{}
\newif\ifFIGS
\FIGSfalse  
\FIGStrue

% \authorcontributions{%Please provide details of author contributions here.
% JL and TL implemented the algorithm, AE and IC interpreted results and worked on mathematical modeling and IC wrote the paper
% }

\tikzexternalenable   
% \pgfplotsset{compat=1.18}
\begin{document}



\begin{figure}
  \tikzexternaldisable
  \tikzsetnextfilename{corrq}
  \centering
  \def\FCOL{Red1}
  \def\AXISCOL{black!10}
  \def\WDT{4in}
  \def\HGT{4in}
  \def\OPC{.2}



\pgfplotsset{
  discard if/.style 2 args={
    x filter/.append code={
      \edef\tempa{\thisrow{#1}}
      \edef\tempb{#2}
      \ifx\tempa\tempb
      \def\pgfmathresult{inf}
      \fi
    }
  },
  discard if not/.style 2 args={
    x filter/.append code={
      \edef\tempa{\thisrow{#1}}
      \edef\tempb{#2}
      \ifx\tempa\tempb
      \else
      \def\pgfmathresult{inf}
      \fi
    }
  },
        % define the style of the `nodes near coords' that should be shown
        % above the point
        nodes near coords above style/.style={
            nodes near coords style={
                anchor=south,
            },
        },
        % define the style of the `nodes near coords' that should be shown
        % below the point
        nodes near coords below style/.style={
            nodes near coords style={
                anchor=north, yshift=-.05in,
            },
        },
        % define the style of the `nodes near coords' that should be shown
        % below the point
        nodes near coords right style/.style={
            nodes near coords style={
                anchor=west,
            },
        },
        % define the style of the `nodes near coords' that should be shown
        % below the point
        nodes near coords null style/.style={
            nodes near coords style={
                anchor=west,text opacity=0,
            },
        },
    }

  
  \begin{tikzpicture}[font=\bf\sffamily\fontsize{5}{5}\selectfont]
    \def\RFILE{Figures/plotdata/fulldataframe.csv}
      \pgfplotstableread[col sep=comma]{\RFILE}\RFILEr;

    \begin{axis}[enlargelimits=false,scale only axis=true,    axis line style={\AXISCOL, opacity=1,ultra  thick, rounded corners=0pt},
      axis on top=false,
      grid style={dashed, thick,gray!40},
      grid=both,
      grid style={dashed, thick,gray!40},
      enlargelimits=0.02, 
      width=\WDT, 
      height=\HGT,
      scaled ticks = false,
      x tick label style={yshift=-.05in,/pgf/number format/fixed,
        /pgf/number format/1000 sep = %\thinspace % Optional if you want to replace comma as the 1000 separator 
      },yticklabel style={/pgf/number format/fixed,
        /pgf/number format/precision=2},,xticklabel style={/pgf/number format/fixed,
        /pgf/number format/precision=2},
      major tick length=0pt,
      yticklabel style={xshift=-.015in}, nodes near coords,
            point meta=explicit symbolic,
            table/meta=strain,
      ]



      \addplot[smooth, ultra thick, opacity=.5,mark=none, nodes near coords null style, ] table[col sep=comma,x=geometric mean of Edistances,y=pred_GM] \RFILE;

      \addplot[nodes near coords null style,forget plot, name path=UB,smooth, ultra thick, mark=none,draw=none ] table[col sep=comma,x=geometric mean of Edistances,y=ub_GM] \RFILE;
      \addplot[nodes near coords null style,forget plot, name path=LB,smooth, ultra thick, mark=none,draw=none ] table[col sep=comma,x=geometric mean of Edistances,y=lb_GM] \RFILE;
      
      \addplot[nodes near coords null style,forget plot,\FCOL,opacity=\OPC] fill between[of=LB and UB];
      
      
      \addplot[only marks, mark=*,mark options={fill=black,scale=1.5},nodes near coords below style, discard if={strain}{A/Ohio/13/2017},discard if={strain}{A/Yunnan/14564/2015},
                discard if={strain}{A/Hong Kong/125/2017},discard if={strain}{A/Netherlands/219/2003} ] table[x=geometric mean of Edistances,y=IRAT Emergence Estimate,col sep=comma] \RFILE;


\pgfplotsinvokeforeach {
                A/Ohio/13/2017,
                A/Netherlands/219/2003,
                A/Yunnan/14564/2015%
            } {
                \addplot+ [only marks, mark=*,mark options={fill=black,scale=1.5},nodes near coords below style,
                    forget plot,
                    nodes near coords above style,
                    discard if not={strain}{#1},
                ] table[x=geometric mean of Edistances,y=IRAT Emergence Estimate,col sep=comma]\RFILE;
            }
\pgfplotsinvokeforeach {
                A/Hong Kong/125/2017%
            } {
                \addplot+ [only marks, mark=*,mark options={fill=black,scale=1.5},nodes near coords below style,
                    forget plot,
                    nodes near coords right style,
                    discard if not={strain}{#1},
                ] table[x=geometric mean of Edistances,y=IRAT Emergence Estimate,col sep=comma]\RFILE;
            }


      
    \end{axis}
  \end{tikzpicture}
  
\captionN{gm corr}\label{figpred0}
\end{figure}






\end{document}
% LocalWords:  Neuraminidase subtype
