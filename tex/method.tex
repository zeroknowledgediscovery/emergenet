\allowdisplaybreaks{
Next, we briefly describe the  details of the computational framework.

\section*{\qnet Framework}

We do not assume that the mutational  variations at the individual indices of a genomic sequence are independent (See Fig~\ref{figscheme}a). Irrespective of whether mutations are truly random~\cite{hernandez2018algorithmically}, since only certain combinations of individual mutations are viable, individual mutations across a genomic sequence replicating in the wild  appear  constrained, which is what is explicitly  modeled in our approach. The mathematical form of our metric is not arbitrary; JS divergence is a symmetricised version of the more common KL divergence~\cite{cover} between distributions, and among  different possibilities, the q-distance  is the simplest metric such that the likelihood of a spontaneous jump (See Eq.~\eqref{fundeq} in Methods) is provably bounded above and below  by simple exponential functions of the q-distance. 

Consider a set of random variables $X=\{X_i\}$, with $i \in \{1, \cdots, N\}$, each taking value from the respective sets $\Sigma_i$. A sample $x \in \prod_1^N \Sigma_i$ is an ordered $N$-tuple, consisting of a realization of each of the variables $X_i$ with the $i^{th}$ entry $x_i$ being the realization of random variable $X_i$. We use the notation $x_{-i}$ and $x^{i,\sigma}$ to denote:
\begin{subequations}\cgather{
x_{-i} \triangleq x_1, \cdots, x_{i-1},x_{i+1},\cdots,x_N\\
x^{i,\sigma} \triangleq x_1, \cdots, x_{i-1},\sigma,x_{i+1},\cdots,x_N, \sigma \in \Sigma_i
}\end{subequations} Also, $\Dx(S)$ denotes the set of probability measures on  a set $S$, $e.g.$,  $\D$ is the set of  distributions on  $\Sigma_i$.

We note that $X$ defines a random field~\cite{vanmarcke2010random} over the index set $\{1, \cdots, N\}$. To clarify the biological picture, we refer to the sample $x$ as an amino acid or nucleotide sequence, identifying the entry at  each index with the corresponding  protein residue or the  nucleotide base pair.

\begin{defn}[\qnet]
For a random field $X=\{X_i\}$ indexed by $i \in \{1, \cdots, N\}$, the \qnet is defined to be the set of predictors $\Phi=\{\qn\}$, $i.e.$, we have:
\cgather{
\qn : \prod_{j \neq i} \Sigma_j \rightarrow \D,
}  where for a sequence $x$, $\Phi_i(x_{-i}) $ estimates the distribution of $X_i$ on the set $\Sigma_i$.
\end{defn}
We use conditional inference trees as models for predictors~\cite{Hothorn06unbiasedrecursive}, although more general models are possible.





\subsection*{Biology-Aware Distance Between Sequences}

\begin{defn}[Q-distance -- pseudo-metric between sequences]\label{defqdistance}
Given two sequences $x,y \in \prod_1^N\Sigma_i$, such that $x,y$ are drawn from the  populations $P,Q$  inducing the \qnet $\Phi^P,\Phi^Q$, respectively,  we define a pseudo-metric $\theta(x,y) $, as follows:
\cgather{\label{q-distance}
\theta(x,y) \triangleq \mathbf{E}_i \left (  \J^{\frac{1}{2}} \left (\qn^P(x_{-i}) , \qn^Q(y_{-i})\right ) \right )
} 
where $ \J(\cdot,\cdot)$ is the Jensen-Shannon divergence~\cite{manning1999foundations} and $\mathbf{E}_i$ indicates expectation over the indices.
\end{defn}
The square-root in the definition arises naturally from the bounds we are able to prove, and is dictated by the form of Pinsker's inequality~\cite{cover}, ensuring that   the sum of the length of successive path fragments equates the length of the path, making it possible to use standard  algorithms  for q-phylogeny construction.





\subsection*{Theoretical Probability Bounds}

The \qnet framework  allows us to rigorously compute bounds on the probability of a spontaneous change of one strain to another, brought about by chance mutations. While any sequence of mutations is equally likely, the ``fitness'' of the resultant strain, or the probability that it will even result in a viable strain, is not. Thus the necessity of preserving  function  dictates that not all random changes  are viable, and the probability of observing some trajectories through the sequence space  are far greater  than others. The \qnet framework allows us to explore this constrained dynamics, as revealed by a sufficiently large set of genomic sequences.

We show in Theorem~\ref{SI-thmbnd} in the supplementary text that at a significance level $\alpha$, with a sequence length $N$, the probability of spontaneous jump of sequence $x$ from population $P$ to sequence $y$ in population $Q$, $Pr(x \rightarrow y)$, is bounded by:
\cgather{\label{fundeq}
\mem{y}^Q e^{ \frac{\sqrt{8}N^2}{1-\alpha}\theta(x,y)} \geqq Pr(x \rightarrow y) \geqq \mem{y}^Q e^{-\frac{\sqrt{8}N^2}{1-\alpha}\theta(x,y)}}
where $\mem{y}^Q$ is the membership probability of strain $y$ in the target population.

The ability to estimate the probability of spontaneous jump between sequences in terms of $\theta$ has  crucial implications. It allows us to 1) construct  a new  phylogeny that directly relates the probability of jumps rather than the number of mutations  between descendants. 2) simulate realistic trajectories in the sequence space from any given initial strain, and 3) estimate drift in the sequence space  by analyzing the statistical characteristics of the diffusion occurring in the strain space.





\subsection*{Application: Predicting Seasonal Strains}

Analyzing the distribution of sequences using the q-distance allows us to estimate seasonal drift, which is particularly applicable to Influenza and Influenza-like viruses for which periodic adjustments of vaccine components are necessary to account for antigenic variations. 

Our prediction is based on the following intuition: since the probability of spontaneous jump to a strain further away in the q-distance is exponentially lower, the q-centroid of the strain distribution (the centroid computed in the q-distance metric) observed over a season is expected to move slowly, and will be close to the dominant strain in the next season. Thus, we estimate the predicted dominant strain $\widehat{x}^{t+1}$ at time $t+1$, as a function of the observed population at time $t$ as follows:
\cgather{
\widehat{x}^{t+1} = \argmin_{x\in P} \sum_{y \in P^t} \theta(x,y)
  }%
where $P^t$ is the sequence population at time $t$ and $P = P^t \cup P^{t-1} \cup P^{t-2} \cup \dots \cup P^1$. Here the unit of time is chosen to reflect the appropriate frequency over which vaccine components are re-assessed. In the case of Influenza, this is typically one year. Using this formulation, we  test if the predicted strains are  closer to the  dominant strain in the classical edit distance, when compared against the WHO vaccine recommendations (See Fig.~\ref{figseasonal}).
}