\xdef\symbolstr{attatggacggatttagac}
\def\iX{10}
\begin{tikzpicture}[anchor=center,scale=\SCALE,text=\TEXTCOL,font=\sffamily\fontsize{7}{7}\selectfont, remember picture]
  % \useasboundingbox (-3.1,3) rectangle (4.75,.5); %all frames having the same size
  %\useasboundingbox (-3.5,-2) rectangle (25,6);
  % \node  [anchor=  west] (Q) at (-3.5,.75) {\BLACKBOX};

  \def\nuPi{3.1459265}
  \pgfmathtruncatemacro{\aaaa}{\iX-1}
  \pgfmathtruncatemacro{\bbbb}{\aaaa-1}
  \foreach \i in {\iX,\aaaa,...,0}{% This one doesn't matter
    \foreach \j in {1,0}{% This will crate a membrane
      % with the front lipids visible
      \pgfmathsetmacro{\dx}{rand*0.0}% A random variance in the x coordinate
      \pgfmathsetmacro{\dy}{rand*0.0}% A random variance in the y coordinate,
      % gives a hight fill to the lipid
      \pgfmathsetmacro{\rot}{rand*0.0}% A random variance in the
      \coordinate (A\i\j) at ({\i+\dx+\rot},{1*\j+\dy+0.6*sin(\i*\nuPi*20)});      
    }
  }
  \foreach \i in {\aaaa,\bbbb,...,1}{
    \StrChar{\symbolstr}{\i}[\symbol]
    \pgfmathtruncatemacro{\iminus}{\i - 1}
    \draw [thin, white,fill=\COLB, opacity=.85] (A\iminus0) -- (A\i0) -- (A\i1) -- (A\iminus1) -- cycle  ;
    \path (A\iminus0) -- (A\iminus1) node[midway] (N1) {};
    \path (A\i0) -- (A\i1) node[midway] (Nx\iX\i) {};
    \path (N1) -- (Nx\iX\i) node[midway] (N) {\bf \texttt \symbol} ;
  }%
  \pgfmathsetmacro{\dx}{\iX /3+.5 }
\end{tikzpicture}
