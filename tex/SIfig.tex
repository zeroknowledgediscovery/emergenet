\setcounter{figure}{0}
\renewcommand{\figurename}{S-Fig.}
\setcounter{table}{0}
\renewcommand{\tablename}{S-Tab.}
%\setcounter{table}{0}
%\renewcommand{\tablename}{SI Tab.}







%#############################################
%#############################################
\ifFIGS
\begin{figure*}[!ht]
  \centering    
  \tikzexternalenable  
  \tikzsetnextfilename{blastvalid}
  %\tikzXtrue
  \iftikzX 
  \input{Figures/sonet}   
  \vspace{-5pt}    
  
  \else 
  \includegraphics[width=0.9\textwidth]{Figures/External/blastvalid.pdf}
  \fi

  \captionN{\textbf{\qdist validation in-silico using Influenza A sequences from NCBI database. Panel a} illustrates that the \enet induced modeling of evolutionary trajectories initiated from known haemagglutinnin (HA)  sequences  are distinct from random paths in the strain space. In particular, random trajectories have more variance, and more importantly, diverge to different regions of the landscape compared to \enet predictions. \textbf{Panels b-e} show that unconstrained Q-sampling  produces sequences maintain a higher degree of similarity to known sequences, as verified by blasting against known HA sequences, have a smaller rate of growth of variance, and produce matches in closer time frames to the initial sequence. \textbf{Panel c} shows that this is not due to simply restricting the  mutational variations, which increases rapidly in both the \enet and the classical metric.}\label{figsoa}
\end{figure*}
\else
\refstepcounter{figure}\label{figsoa}
\fi
%#############################################
%#############################################


%#############################################
%#############################################
\begin{table*}[!ht]
\def\ACOL{teal!30}
\def\BCOL{Red1!30}
\def\CCOL{gray!30}
\captionN{Examples: \enet induced distance varying for fixed sequence pair when background population changes (rows 1 -5), sequences with small edit distance and large q-distance, and the converse (rows 6-9)}\label{tabex}
\begin{tabular}{L{.1in}|L{.45in}|L{1.85in}|L{2in}|L{.5in}|L{.325in}|L{.325in}}\hline
& Edit dist. & Sequence A & Sequence B & \enet E-dist. & Year A$^\star$ & Year B$^\star$\\\hline
\rowcolor{\ACOL}1&18 & A/Singapore/23J/2007 & A/Tennessee/UR06-0294/2007 & 0.0111 & 2007 & 2007\\\hline
\rowcolor{\ACOL}2&18 & A/Singapore/23J/2007 & A/Tennessee/UR06-0294/2007 & 0.0094 & 2008 & 2008\\\hline
\rowcolor{\ACOL}3&18 & A/Singapore/23J/2007 & A/Tennessee/UR06-0294/2007 & 0.0027 & 2009 & 2009\\\hline
\rowcolor{\ACOL}4&18 & A/Singapore/23J/2007 & A/Tennessee/UR06-0294/2007 & 0.0025 & 2010 & 2010\\\hline
\rowcolor{\ACOL}5&18 & A/Singapore/23J/2007 & A/Tennessee/UR06-0294/2007 & 0.6163 & 2007 & 2010\\\hline
\rowcolor{\BCOL}6&11 & A/Naypyitaw/M783/2008 & A/Singapore/201/2008 &      0.8852 & 2008 & 2008\\\hline
\rowcolor{\BCOL}7&15 & A/Cambodia/W0908339/2012 & A/Singapore/DMS1233/2012&0.2737 & 2012 & 2012\\\hline
\rowcolor{\CCOL}8&126 & A/South Dakota/03/2008 & A/Singapore/10/2008 &     0.3034 & 2008 & 2008\\\hline
\rowcolor{\CCOL}9&141 & A/Jodhpur/3248/2012 & A/Cambodia/W0908339/2012 &   0.2405 & 2012 & 2012\\\hline
\end{tabular}
\flushleft
$^\star$Year A and year B correspond to the assumed collection years for sequences A and B respectively for the purpose of this example. Sequence A in row 1 is collected in 2007, but is assumed to be from different years in rows 2-4 to demonstrate the change in q-distance from sequence B, arising only from a change in the background population.
\end{table*}
%#############################################
%#############################################




%#############################################
%#############################################
\begin{table*}[!ht]
 \mnp{2.75in}{ \centering
 \captionN{Correlation between \qdist and edit distance between sequence pairs}\label{tabcor}
\sffamily\fontsize{8}{8}\selectfont
\begin{tabular}{L{1.35in}|L{.65in}}\hline
Phenotypes & Correlation \\\hline
 Influenza H1N1 HA  &0.76\\\hline
 Influenza H1N1 NA &0.74\\\hline
 Influenza H3N2 HA &0.85\\\hline
 Influenza H3N2 NA &0.79\\\hline
% \hcov &0.52\\\hline
\end{tabular}
}\hfill
\mnp{3.75in}{  \centering
 \captionN{Number of sequences collected from public databases}\label{tabseq}
\sffamily\fontsize{8}{8}\selectfont
\begin{tabular}{L{.7in}|L{1.55in}|L{.95in}}\hline
Database & Strain & No. of Sequences \\\hline
NCBI& Influenza  H1N1  HA &17,894\\\hline
NCBI& Influenza  H1N1  NA &16,637\\\hline
NCBI& Influenza  H3N2  HA &18,265\\\hline
NCBI& Influenza  H3N2  NA &14,699\\\hline
GISAID& Influenza  H1N1  HA &1,528\\\hline
GISAID& Influenza  H1N1  NA &1,490\\\hline
GISAID& Influenza  H3N2  HA &13,975\\\hline
GISAID& Influenza  H3N2  NA &13,811\\\hline
Total & &98,299\\\hline
\end{tabular}
}
\end{table*}
%#############################################
%#############################################


%#############################################
%#############################################
\ifFIGS
\begin{figure*}[!ht]
  \centering
  \tikzexternalenable
  \tikzsetnextfilename{dom}

  \iftikzX
  
\begin{tikzpicture}[font=\bf\sffamily\fontsize{8}{8}\selectfont]
\def\DATAA{Figures/plotdata/h1n1humanHA_dom_ldist.dat}
\def\DATAB{Figures/plotdata/h1n1humanNA_dom_ldist.dat}
\def\DATAC{Figures/plotdata/h3n2humanHA_dom_ldist.dat}
\def\DATAD{Figures/plotdata/h3n2humanNA_dom_ldist.dat}

\node[] (A) at (0,0) {
\begin{tikzpicture}
\begin{groupplot}[group style={
        group size=2 by 2,
        xlabels at=edge bottom,
        xticklabels at=edge bottom,
        vertical sep=0.2in, horizontal sep=.35in,
    },height=2in,width=2in]

\nextgroupplot[ylabel=probability,ylabel style={xshift=-.8in,yshift=-.1in}]
\addplot [
    hist={
        bins=50,
        data min=0,
        data max=30,
density=true
    }  , fill=black!50
] table [y index=0] {\DATAA};
\addlegendentry{HA H1N1}


\nextgroupplot[]
\addplot [
    hist={
        bins=50,
        data min=0,
        data max=30,
density=true
    }    , fill=black!50
] table [y index=0] {\DATAB};
\addlegendentry{NA H1N1}


\nextgroupplot[xlabel=Edit Distance from Dominant Strain,xlabel style={xshift=.8in}]
\addplot [
    hist={
        bins=50,
        data min=0,
        data max=30,
density=true
    }    , fill=black!50
] table [y index=0] {\DATAC};
\addlegendentry{HA H3N2}


\nextgroupplot[]
\addplot [
    hist={
        bins=50,
        data min=0,
        data max=30,
density=true
    }    , fill=black!50
] table [y index=0] {\DATAD};
\addlegendentry{NA H3N2}


\end{groupplot}
\end{tikzpicture}
};

\node [anchor=south west] (L1) at (A.north west) {{\large a.} Distribution around dominant strain};

\end{tikzpicture}

  
  \else
  \includegraphics[width=.6\textwidth]{Figures/External/dom.pdf}
  \vspace{0pt}
  \fi
  
\vspace{0pt}

\captionN{\textbf{No. of mutations from the seasonal dominant strain over the years} The quasispecies that circulates each season for each sub-type is tightly distributed around the dominant strain on average.}\label{figdom}
\end{figure*}
\else
\refstepcounter{figure}\label{figdom}
\fi
%#############################################
%#############################################





% \begin{table}[!ht]\centering
% \captionN{H1N1 NA Northern Hemisphere}\label{tabrec2}

% \sffamily\fontsize{7}{8}\selectfont

% \begin{tabular}{L{.37in}|L{1.62in}|L{1.62in}|L{1.62in}|L{.25in}|L{.25in}}\hline
Year & WHO Recommendation & Dominant Strain & \qnet Recommendation & WHO Error & \qnet Error \\\hline
2001-02& A/New  Caledonia/20/99 & A/New  York/447/2001 & A/Memphis/15/2000 &4&4\\\hline
2002-03& A/New  Caledonia/20/99 & A/Paris/0833/2002 & A/New  York/341/2001 &1&5\\\hline
2003-04& A/New  Caledonia/20/99 & A/Memphis/5/2003 & A/New  York/291/2002 &3&5\\\hline
2004-05& A/New  Caledonia/20/99 & A/Singapore/14/2004 & A/New  York/223/2003 &2&3\\\hline
2005-06& A/New  Caledonia/20/99 & A/Taiwan/5524/2005 & A/Florida/3e/2004 &3&0\\\hline
2006-07& A/New  Caledonia/20/99 & A/Massachusetts/08/2006 & A/Sofia/361/2005 &4&2\\\hline
2007-08& A/Solomon  Islands/3/2006 & A/Tennessee/UR06-0106/2007 & A/Sofia/490/2006 &9&2\\\hline
2008-09& A/Brisbane/59/2007 & A/Sendai/TU66/2008 & A/Maryland/04/2007 &0&3\\\hline
2009-10& A/Brisbane/59/2007 & A/Thailand/SR08021/2009 & A/Paris/910/2008 &87&87\\\hline
2010-11& A/California/7/2009 & A/Finland/2460N/2010 & A/Rome/709/2009 &2&9\\\hline
2011-12& A/California/7/2009 & A/Tula/CRIE-GSYu/2011 & A/Oman/SQUH-40/2010 &4&2\\\hline
2012-13& A/California/7/2009 & A/Bangalore/697-32/2012 & A/Nizhnii  Novgorod/CRIE-ZCA/2011 &4&0\\\hline
2013-14& A/California/7/2009 & A/Jiangsugusu/SWL1824/2013 & A/LongYan/SWL33/2013 &5&3\\\hline
2014-15& A/California/7/2009 & A/LongYan/SWL2457/2014 & A/Utah/06/2013 &9&3\\\hline
2015-16& A/California/7/2009 & A/Michigan/45/2015 & A/Maryland/02/2014 &14&4\\\hline
2016-17& A/California/7/2009 & A/Mexico/4436/2016 & A/India/Pun151245/2015 &14&0\\\hline
2017-18& A/Michigan/45/2015 & A/Illinois/37/2017 & A/Utah/02/2016 &3&3\\\hline
2018-19& A/Michigan/45/2015 & A/Kenya/47/2018 & A/Maine/24/2017 &4&0\\\hline
2019-20& A/Brisbane/02/2018 & A/Texas/7939/2019 & A/Missouri/03/2018 &1&0\\\hline
2020-21& A/Hawaii/70/2019 &A/Togo/897/2020& A/Texas/112/2019 &0&5\\\hline
2021-22& A/Victoria/2570/2019 &A/Cote\_d'Ivoire/3729/2021&A/Togo/0071/2021&1&5\\\hline
2022-23& -1 &-1& A/Lyon/820/2021 &-1&-1\\\hline
\end{tabular}

% \flushleft

% \fontsize{7}{7}\selectfont
% $^\star$ Dominant strain is calculated as the one closest to the centroid in the strain space that year in the edit distance metric
% \end{table}
% %#############################################
% %#############################################

% \begin{table}[!ht]\centering
% \captionN{H1N1 NA Southern Hemisphere}\label{tabrec3}

% \sffamily\fontsize{7}{8}\selectfont

% \begin{tabular}{L{.37in}|L{1.62in}|L{1.62in}|L{1.62in}|L{.25in}|L{.25in}}\hline
Year & WHO Recommendation & Dominant Strain & \qnet Recommendation & WHO Error & \qnet Error \\\hline
2001-02& A/New  Caledonia/20/99 & A/New  York/447/2001 & A/Canterbury/37/2000 &4&6\\\hline
2002-03& A/New  Caledonia/20/99 & A/Paris/0833/2002 & A/New  York/447/2001 &1&5\\\hline
2003-04& A/New  Caledonia/20/99 & A/Memphis/5/2003 & A/New  York/291/2002 &3&5\\\hline
2004-05& A/New  Caledonia/20/99 & A/Singapore/14/2004 & A/Memphis/5/2003 &2&3\\\hline
2005-06& A/New  Caledonia/20/99 & A/Taiwan/5524/2005 & A/Canterbury/106/2004 &3&6\\\hline
2006-07& A/New  Caledonia/20/99 & A/Massachusetts/08/2006 & A/Sofia/361/2005 &4&2\\\hline
2007-08& A/New  Caledonia/20/99 & A/Tennessee/UR06-0106/2007 & A/Thailand/RMSC-UDN-20/2006 &4&8\\\hline
2008-09& A/Solomon  Islands/3/2006 & A/Sendai/TU66/2008 & A/Tennessee/UR06-0151/2007 &15&13\\\hline
2009-10& A/Brisbane/59/2007 & A/Thailand/SR08021/2009 & A/Nebraska/07/2008 &87&87\\\hline
2010-11& A/California/7/2009 & A/Finland/2460N/2010 & A/Rome/709/2009 &2&9\\\hline
2011-12& A/California/7/2009 & A/Tula/CRIE-GSYu/2011 & A/Finland/2460N/2010 &4&2\\\hline
2012-13& A/California/7/2009 & A/Bangalore/697-32/2012 & A/Tula/CRIE-GSYu/2011 &4&0\\\hline
2013-14& A/California/7/2009 & A/Jiangsugusu/SWL1824/2013 & A/Oman/SQUH-63/2012 &5&4\\\hline
2014-15& A/California/7/2009 & A/LongYan/SWL2457/2014 & A/NanPing/SWL1640/2013 &9&6\\\hline
2015-16& A/California/7/2009 & A/Michigan/45/2015 & A/LongYan/SWL2457/2014 &14&5\\\hline
2016-17& A/California/7/2009 & A/Mexico/4436/2016 & A/Michigan/45/2015 &14&0\\\hline
2017-18& A/Michigan/45/2015 & A/Illinois/37/2017 & A/Mexico/4436/2016 &3&3\\\hline
2018-19& A/Michigan/45/2015 & A/Kenya/47/2018 & A/Kentucky/26/2017 &4&2\\\hline
2019-20& A/Michigan/45/2015 & A/Texas/7939/2019 & A/Kenya/47/2018 &4&0\\\hline
2020-21& A/Brisbane/02/2018 &A/Togo/897/2020& A/Texas/7939/2019 &6&5\\\hline
2021-22& A/Victoria/2570/2019 &A/Cote\_D'Ivoire/1496/2021& A/NAGASAKI/8/2020 &1&6\\\hline
2022-23& -1 &-1& A/Dakar/35/2021 &-1&-1\\\hline
\end{tabular}

% \flushleft

% \fontsize{7}{7}\selectfont
% $^\star$ Dominant strain is calculated as the one closest to the centroid in the strain space that year in the edit distance metric
% \end{table}


% \begin{table}[!ht]\centering
% \captionN{H3N2 NA Northern Hemisphere}\label{tabrec6}

% \sffamily\fontsize{7}{8}\selectfont

% \begin{tabular}{L{.37in}|L{1.62in}|L{1.62in}|L{1.62in}|L{.25in}|L{.25in}}\hline
Year & WHO Recommendation & Dominant Strain & \qnet Recommendation & WHO Error & \qnet Error \\\hline
2003-04&A/Moscow/10/99& A/Denmark/107/2003 & A/New  York/100/2002 &13&3\\\hline
2004-05& A/Fujian/411/2002 &A/Hyogo/36/2004& A/New  York/20/2003 &3&16\\\hline
2005-06& A/California/7/2004 & A/Denmark/203/2005 & A/Hong  Kong/HKU20/2004 &4&0\\\hline
2006-07& A/Wisconsin/67/2005 & A/Berlin/32/2006 & A/Mexico/InDRE2227/2005 &1&1\\\hline
2007-08& A/Wisconsin/67/2005 & A/Brazil/80/2007 & A/Baden-Wuerttemberg/17/2006 &8&7\\\hline
2008-09& A/Brisbane/10/2007 & A/Missouri/05/2008 & A/Washington/01/2007 &3&2\\\hline
2009-10& A/Brisbane/10/2007 & A/Oklahoma/09/2009 & A/Wisconsin/24/2008 &3&1\\\hline
2010-11& A/Perth/16/2009 & A/California/17/2010 & A/New  York/70/2009 &2&3\\\hline
2011-12& A/Perth/16/2009 & A/Texas/14/2011 & A/California/14/2010 &3&2\\\hline
2012-13& A/Victoria/361/2011 & A/New  York/02/2012 & A/Singapore/C2011.493/2011 &4&1\\\hline
2013-14& A/Victoria/361/2011 & A/Michigan/02/2013 & A/New  York/01/2012 &3&1\\\hline
2014-15& A/Texas/50/2012 & A/Tehran/69634/2014 & A/Boston/DOA2-176/2013 &3&1\\\hline
2015-16& A/Switzerland/9715293/2013 &A/Parma/471/2015& A/Thailand/CU-B10520/2014 &3&0\\\hline
2016-17& A/Hong  Kong/4801/2014 & A/North  Carolina/62/2016 & A/Delaware/02/2015 &7&2\\\hline
2017-18& A/Hong  Kong/4801/2014 & A/Texas/277/2017 & A/New  York/03/2016 &8&0\\\hline
2018-19& A/Singapore/INFIMH-16-0019/2016 & A/Japan/NHRC\_FDX70352/2018 & A/Colorado/11/2017 &4&3\\\hline
2019-20& A/Kansas/14/2017 & A/Washington/9757/2019 &A/Guangxi-Fangcheng/54/2019&3&11\\\hline
2020-21& A/Hong  Kong/2671/2019 & A/Bangladesh/1004005/2020 & A/Maryland/02/2019 &3&13\\\hline
2021-22& A/Cambodia/e0826360/2020 &A/Stockholm/10/2022& A/Bangladesh/1916/2020 &2&2\\\hline
2022-23& -1 &-1& A/Iowa/20/2022 &-1&-1\\\hline
\end{tabular}

% \flushleft

% \fontsize{8}{8}\selectfont
% $^\star$ Dominant strain is calculated as the one closest to the centroid in the strain space that year in the edit distance metric
% \end{table}
% \vspace{100pt}
% %#############################################
% %#############################################

% \begin{table}[!ht]\centering
% \captionN{H3N2 NA Southern Hemisphere}\label{tabrec7}

% \sffamily\fontsize{7}{8}\selectfont

% \begin{tabular}{L{.37in}|L{1.62in}|L{1.62in}|L{1.62in}|L{.25in}|L{.25in}}\hline
Year & WHO Recommendation & Dominant Strain & \qnet Recommendation & WHO Error & \qnet Error \\\hline
2003-04&A/Moscow/10/99& A/Denmark/107/2003 & A/New  York/101/2002 &13&3\\\hline
2004-05& A/Fujian/411/2002 &A/Hyogo/36/2004& A/New  York/20/2003 &3&16\\\hline
2005-06& A/Wellington/1/2004 & A/Denmark/203/2005 & A/Wellington/1/2004 &2&2\\\hline
2006-07& A/California/7/2004 & A/Berlin/32/2006 & A/Mexico/InDRE2227/2005 &3&1\\\hline
2007-08& A/Wisconsin/67/2005 & A/Brazil/80/2007 & A/Ohio/06/2006 &8&10\\\hline
2008-09& A/Brisbane/10/2007 & A/Missouri/05/2008 & A/Brazil/80/2007 &3&2\\\hline
2009-10& A/Brisbane/10/2007 & A/Oklahoma/09/2009 & A/Wisconsin/24/2008 &3&1\\\hline
2010-11& A/Perth/16/2009 & A/California/17/2010 & A/New  York/70/2009 &2&3\\\hline
2011-12& A/Perth/16/2009 & A/Texas/14/2011 & A/Virginia/05/2010 &3&2\\\hline
2012-13& A/Perth/16/2009 & A/New  York/02/2012 & A/Texas/14/2011 &4&1\\\hline
2013-14& A/Victoria/361/2011 & A/Michigan/02/2013 & A/New  York/02/2012 &3&3\\\hline
2014-15& A/Texas/50/2012 & A/Tehran/69634/2014 & A/Michigan/02/2013 &3&1\\\hline
2015-16& A/Switzerland/9715293/2013 &A/Parma/471/2015& A/Tehran/69634/2014 &3&2\\\hline
2016-17& A/Hong  Kong/4801/2014 & A/North  Carolina/62/2016 &A/Parma/471/2015&7&2\\\hline
2017-18& A/Hong  Kong/4801/2014 & A/Texas/277/2017 & A/Guangdong/264/2016 &8&0\\\hline
2018-19& A/Singapore/INFIMH-16-0019/2016 & A/Japan/NHRC\_FDX70352/2018 & A/Texas/277/2017 &4&3\\\hline
2019-20& A/Switzerland/8060/2017 & A/Washington/9757/2019 & A/Pennsylvania/317/2018 &10&10\\\hline
2020-21& A/South  Australia/34/2019 & A/Bangladesh/1004005/2020 & A/Washington/9757/2019 &1&13\\\hline
2021-22& A/Hong Kong/2671/2019 &A/India/PUN-NIV301718/2021	& A/Darwin/11/2021 &6&1\\\hline
2022-23& -1 &-1& A/Texas/12723/2022 &-1&-1\\\hline
\end{tabular}

% \flushleft

% \fontsize{7}{7}\selectfont
% $^\star$ Dominant strain is calculated as the one closest to the centroid in the strain space that year in the edit distance metric
% \end{table}
% %#############################################
% %#############################################





%#############################################
%#############################################

\begin{table}\centering
\captionN{H1N1 NA Northern Hemisphere (Multi-cluster)}\label{tabrec8}

\sffamily\fontsize{7}{8}\selectfont

\input{Figures/tabdata/north_h1n1_na_3cluster.tex}
\flushleft

\fontsize{7}{7}\selectfont
$^\star$ Dominant strain is calculated as the one closest to the centroid in the strain space that year in the edit distance metric
\end{table}

%#############################################
%#############################################

\begin{table}\centering
\captionN{H1N1 NA Southern Hemisphere (Multi-cluster)}\label{tabrec9}

\sffamily\fontsize{7}{8}\selectfont

\begin{tabular}{L{.37in}|L{1.33in}|L{.25in}|L{.25in}|L{.25in}|L{1.65in}|L{1.65in}}\hline
Year & WHO Recommendation & WHO Error & \qnet Error 1 & \qnet Error 2 & \qnet Recommendation 1 & \qnet  Recommendation 2 \\\hline
2001-02& A/New  Caledonia/20/99 &4&1&6& A/New  South  Wales/26/2000 & A/Canterbury/37/2000 \\\hline
2002-03& A/New  Caledonia/20/99 &1&0&5& A/Wellington/1/2001 & A/New  York/447/2001 \\\hline
2003-04& A/New  Caledonia/20/99 &3&2&8& A/Paris/0833/2002 & A/Taiwan/141/2002 \\\hline
2004-05& A/New  Caledonia/20/99 &2&3&4& A/Memphis/5/2003 & A/Hanoi/1004/2003 \\\hline
2005-06& A/New  Caledonia/20/99 &3&0&1& A/Denmark/130/2004 & A/Paris/650/2004 \\\hline
2006-07& A/New  Caledonia/20/99 &4&2&8& A/Sofia/361/2005 & A/Wellington/11/2005 \\\hline
2007-08& A/New  Caledonia/20/99 &4&4&8& A/Sofia/246/2006 & A/New  York/8/2006 \\\hline
2008-09& A/Solomon  Islands/3/2006 &15&13&19& A/Tennessee/UR06-0151/2007 & A/Ohio/UR06-0178/2007 \\\hline
2009-10& A/Brisbane/59/2007 &87&88&90& A/Sendai/TU66/2008 & A/Japan/618/2008 \\\hline
2010-11& A/California/7/2009 &2&1&6& A/South  Carolina/WRAIR1645P/2009 & A/Wisconsin/629-D00809/2009 \\\hline
2011-12& A/California/7/2009 &4&1&3& A/England/21680633/2010 & A/Hangzhou/178/2010 \\\hline
2012-13& A/California/7/2009 &4&1&22& A/Joshkar-Ola/CRIE-BLP/2011 & A/Rio Grande do Sul/578/2011 \\\hline
2013-14& A/California/7/2009 &5&4&13& A/Thailand/MR10580/2012 & A/Mexico/INMEGEN-INER  15/2012 \\\hline
2014-15& A/California/7/2009 &9&3&7& A/Minnesota/02/2013 & A/Helsinki/430/2013 \\\hline
2015-16& A/California/7/2009 &14&4&7& A/Helsinki/808M/2014 & A/Virginia/NHRC430739/2014 \\\hline
2016-17& A/California/7/2009 &14&0&3& A/Michigan/45/2015 & A/Colorado/30/2015 \\\hline
2017-18& A/Michigan/45/2015 &3&3&8& A/Mexico/4436/2016 & A/Arizona/03/2016 \\\hline
2018-19& A/Michigan/45/2015 &4&0&4& A/California/NHRC\_QV11073/2017 & A/Minnesota/35/2017 \\\hline
2019-20& A/Michigan/45/2015 &4&0&2& A/Kenya/47/2018 & A/Colorado/7682/2018 \\\hline
2020-21& A/Brisbane/02/2018 &5&2&7& A/California/NHRC-OID\_BOX-ILI-0012/2019 & A/Indiana/30/2019 \\\hline
2021-22& A/Victoria/2570/2019 &1&7&58& A/Togo/0155/2021 & A/Shandong/00204/2021 \\\hline
2022-23& -1 &-1&-1&-1& A/Switzerland/86136/2022 & A/Wisconsin/04/2021 \\\hline
\end{tabular}
\flushleft

\fontsize{7}{7}\selectfont
$^\star$ Dominant strain is calculated as the one closest to the centroid in the strain space that year in the edit distance metric
\end{table}

%#############################################
%#############################################

\begin{table}\centering
\captionN{H3N2 NA Northern Hemisphere (Multi-cluster)}\label{tabrec10}

\sffamily\fontsize{7}{8}\selectfont

\input{Figures/tabdata/north_h3n2_na_3cluster.tex}
\flushleft

\fontsize{7}{7}\selectfont
$^\star$ Dominant strain is calculated as the one closest to the centroid in the strain space that year in the edit distance metric
\end{table}

%#############################################
%#############################################

\begin{table}\centering
\captionN{H3N2 NA Southern Hemisphere (Multi-cluster)}\label{tabrec11}

\sffamily\fontsize{7}{8}\selectfont

\begin{tabular}{L{.37in}|L{1.33in}|L{.25in}|L{.25in}|L{.25in}|L{1.65in}|L{1.65in}}\hline
Year & WHO Recommendation & WHO Error & \qnet Error 1 & \qnet Error 2 & \qnet Recommendation 1 & \qnet  Recommendation 2 \\\hline
2003-04&A/Moscow/10/99&13&4&5& A/Auckland/612/2002 & A/New  York/87/2002 \\\hline
2004-05& A/Fujian/411/2002 &3&16&18& A/New  York/20/2003 & A/New  York/12/2003 \\\hline
2005-06& A/Wellington/1/2004 &2&1&7& A/New  York/358/2004 & A/Singapore/36/2004 \\\hline
2006-07& A/California/7/2004 &3&3&8&A/Macau/557/2005& A/Hong  Kong/HKU53/2005 \\\hline
2007-08& A/Wisconsin/67/2005 &8&0&10& A/Wisconsin/42/2006 & A/Wisconsin/44/2006 \\\hline
2008-09& A/Brisbane/10/2007 &3&4&10& A/Missouri/06/2007 & A/Japan/72/2007 \\\hline
2009-10& A/Brisbane/10/2007 &3&1&7& A/Wisconsin/24/2008 & A/Mississippi/UR07-0042/2008 \\\hline
2010-11& A/Perth/16/2009 &2&3&8& A/New  York/70/2009 & A/Japan/883/2009 \\\hline
2011-12& A/Perth/16/2009 &3&2&2& A/California/19/2010 & A/Virginia/05/2010 \\\hline
2012-13& A/Perth/16/2009 &4&1&12& A/Texas/14/2011 & A/Singapore/GP1684/2011 \\\hline
2013-14& A/Victoria/361/2011 &3&1&5& A/Idaho/38/2012 & A/Pavia/135/2012 \\\hline
2014-15& A/Texas/50/2012 &3&1&1& A/Nevada/05/2013 & A/Michigan/02/2013 \\\hline
2015-16& A/Switzerland/9715293/2013 &3&0&4& A/Nicaragua/6866\_14/2014 & A/Iran/91244/2014 \\\hline
2016-17& A/Hong Kong/4801/2014 &7&1&25& A/New  Jersey/13/2015 & A/California/NHRC\_BRD41056N/2015 \\\hline
2017-18& A/Hong Kong/4801/2014 &9&1&4& A/Guangdong/264/2016 & A/Victoria/668/2016 \\\hline
2018-19& A/Singapore/INFIMH-16-0019/2016 &3&2&4& A/Netherlands/3530/2017 & A/Washington/17/2017 \\\hline
2019-20& A/Switzerland/8060/2017 &10&4&10& A/England/538/2018 & A/California/BRD12490N/2018 \\\hline
2020-21& A/South  Australia/34/2019 &1&1&13& A/England/9738/2019 & A/Washington/9757/2019 \\\hline
2021-22& A/Hong Kong/2671/2019 &6&1&49& A/Darwin/11/2021 & A/Hawaii/28/2020 \\\hline
2022-23& -1 &-1&-1&-1& A/Congo/313/2021	 & A/Texas/12723/2022 \\\hline
\end{tabular}
\flushleft

\fontsize{7}{7}\selectfont
$^\star$ Dominant strain is calculated as the one closest to the centroid in the strain space that year in the edit distance metric
\end{table}
%#############################################
% %#############################################
% %#############################################
% \ifFIGS
% \begin{figure*}[!t]
%   \centering    
%   %\tikzexternalenable  
%   %\tikzsetnextfilename{IRAT_split.png}
%   %\iftikzX 
%   %\vspace{-5pt}    
  
%   %\else 
%   \includegraphics[width=0.9\textwidth]{Figures/External/IRAT_split.png}
%   %\fi

%   \captionN{\textbf{IRAT vs. Q-distance relationship for H1- and H3- sub-types, using past year data vs. using all data.} On the result when computing average q-distance between the target strain and the circulating human strains from the past year, and on the right is the result when using all available human strains of that sub-type. Evidently, the former has a much higher correlation, since a strain being ``close" to humans at some point does not necessarily mean being close now.}\label{irat}
% \end{figure*}
% \else
% \refstepcounter{figure}\label{irat}
% \fi
% %#############################################
% %#############################################





%#############################################
%#############################################

\begin{table}[!ht]\centering
\captionN{Influenza A Strains Evaluated by IRAT and Corresponding \enet Computed Risk Scores}\label{irattab}

\sffamily\fontsize{7}{8}\selectfont

\begin{tabular}{L{1.32in}|L{.35in}|L{.3in}|L{.3in}|L{.3in}|L{.35in}|L{.35in}|L{.35in}|L{.35in}|L{.32in}|L{.3in}|L{.3in}}\hline
Influenza Virus & Subype & IRAT Date &IRAT Emergence Score &IRAT Impact Score &HA Enet Sample &NA Enet Sample &HA Avg. E-dist. &NA Avg. E-dist. &Geom. Mean&Enet Emergence Score&Enet Impact Score \\\hline
 A/swine/Shandong/1207/2016 &H1N1& Jul  2020 &7.5&6.9&1000&1000&0.0941&0.0205&0.0440&6.00&6.06\\\hline
 A/Ohio/13/2017 &H3N2& Jul  2019 &6.6&5.8&1000&1000&0.0184&0.0306&0.0238&6.34&6.34\\\hline
 A/Hong  Kong/125/2017 &H7N9& May  2017 &6.5&7.5&437&437&0.0296&0.0058&0.0131&6.54&6.49\\\hline
 A/Shanghai/02/2013 &H7N9& Apr  2016 &6.4&7.2&178&178&0.0055&0.0036&0.0044&6.70&6.62\\\hline
 A/Anhui-Lujiang/39/2018 &H9N2& Jul  2019 &6.2&5.9&31&30&0.0290&0.1681&0.0698&5.58&5.74\\\hline
 A/Indiana/08/2011 &H3N2& Dec  2012 &6.0&4.5&1000&1000&0.0523&0.0091&0.0218&6.37&6.36\\\hline
 A/California/62/2018 &H1N2& Jul  2019 &5.8&5.7&55&55&0.1089&0.0610&0.0815&5.41&5.61\\\hline
 A/Bangladesh/0994/2011$^{\star\star\star}$ &H9N2& Feb  2014 &5.6&5.4&-1&-1&0.2078&0.1823&0.1947&4.21&4.69\\\hline
 A/Sichuan/06681/2021 &H5N6& Oct  2021 &5.3&6.3&45&45&0.3616&0.0518&0.1369&4.72&5.07\\\hline
 A/Vietnam/1203/2004 &H5N1& Nov  2011 &5.2&6.6&258&246&0.1673&0.0111&0.0430&6.01&6.07\\\hline
 A/Yunnan/14564/2015$^{\star\star}$ &H5N6& Apr  2016 &5.0&6.6&344&331&0.3482&0.2987&0.3225&3.80&4.45\\\hline
 A/Astrakhan/3212/2020$^{\star\star}$ &H5N8& Mar  2021 &4.6&5.2&381&365&0.1603&0.3472&0.2359&3.97&4.53\\\hline
 A/Netherlands/219/2003 &H7N7& Jun  2012 &4.6&5.8&46&46&0.2757&0.3521&0.3115&3.80&4.45\\\hline
 A/American  wigeon/South  Carolina/AH0195145/2021 &H5N1& Mar  2022 &4.4&5.1&335&323&0.1722&0.5114&0.2967&3.81&4.45\\\hline
 A/Jiangxi-Donghu/346/2013$^{\star\star\star}$ &H10N8& Feb  2014 &4.3&6.0&-1&-1&0.2088&0.2101&0.2094&4.11&4.62\\\hline
 A/gyrfalcon/Washington/ 41088/2014$^{\star\star}$ &H5N8& Mar  2015 &4.2&4.6&341&328&0.1532&0.3424&0.2290&4.00&4.55\\\hline
 A/Northern  pintail/ Washington/40964/2014$^{\star\star}$ &H5N2& Mar  2015 &3.8&4.1&341&328&0.1529&0.3799&0.2410&3.95&4.51\\\hline
 A/canine/Illinois/12191/2015 &H3N2& Jun  2016 &3.7&3.7&1000&1000&0.0607&0.1509&0.0957&5.22&5.45\\\hline
 A/American  green-winged  teal /Washington/1957050/2014 &H5N1& Mar 2015 &3.6&4.1&326&314&0.1911&0.4482&0.2927&3.82&4.45\\\hline
 A/turkey/Indiana/1573-2/2016$^{\star\star}$ &H7N8& Jul  2017 &3.4&3.9&495&494&0.1130&0.7738&0.2957&3.81&4.45\\\hline
 A/chicken/Tennessee/17-007431-3/2017 &H7N9& Oct  2017 &3.1&3.5&496&495&0.1027&0.2569&0.1624&4.47&4.88\\\hline
 A/chicken/Tennessee/17-007147-2/2017 &H7N9& Oct  2017 &2.8&3.5&496&495&0.2095&0.2541&0.2307&3.99&4.54\\\hline
 A/duck/New  York/1996 $^\star$&H1N1& Nov  2011 &2.3&2.4&1000&1000&-1&-1&-1&-1&-1\\\hline
 \end{tabular}
\flushleft

\fontsize{8}{8}\selectfont
$^\star$ HA strain is not available for A/duck/New York/1996, so this strain is omitted.\\
$^{\star\star}$ Could not construct a \enet of human sequence data available for that virus sub-type (less than 30 strains), so we constructed a \enet using all human strains that match the HA sub-type, i.e. H5NX for H5N6.\\
$^{\star\star\star}$ These strains did not have enough human sequence data to generate a \enet, even when only considering the HA sub-type. Thus, we estimated the risk score using every \enet from the other IRAT strains, and took the average among NA and HA. Finally, we took the geometric mean of the resulting NA and HA averages.
\end{table}

%#############################################


%#############################################
\ifFIGS
%#############################################

\begin{table}[!ht]\centering
\captionN{Influenza A Strains Evaluated by IRAT and Corresponding \enet Computed Current Risk Scores}\label{irattab_current}

\sffamily\fontsize{7}{8}\selectfont

\begin{tabular}{L{1.25in}|L{.35in}|L{.3in}|L{.3in}|L{.3in}|L{.35in}|L{.35in}|L{.35in}|L{.35in}|L{.32in}|L{.3in}|L{.3in}}\hline
Influenza Virus & Subype & IRAT Date &IRAT Emergence Score &IRAT Impact Score &HA Sample &NA Sample &HA \erisk & NA \erisk &Geom. Mean&\qnet Emergence Score&\qnet Impact Score \\\hline
 A/swine/Shandong/1207/2016 &H1N1& Jul  2020 &7.5&6.9&1000&1000&-0.0599&-0.0417&0.0500&5.8&5.8\\\hline
 A/Ohio/13/2017 &H3N2& Jul  2019 &6.6&5.8&1000&1000&-0.0091&-0.0692&0.0251&6.2&6.0\\\hline
 A/Hong  Kong/125/2017 &H7N9& May  2017 &6.5&7.5&1000&1000&-0.0092&-0.0046&0.0065&6.7&6.6\\\hline
 A/Shanghai/02/2013 &H7N9& Apr  2016 &6.4&7.2&1000&1000&-0.0031&-0.0044&0.0037&6.8&6.6\\\hline
 A/Anhui-Lujiang/39/2018 &H9N2& Jul  2019 &6.2&5.9&58&58&-0.0157&-0.0467&0.0271&6.2&6.0\\\hline
 A/Indiana/08/2011 &H3N2& Dec  2012 &6.0&4.5&1000&1000&-0.0176&-0.0184&0.0180&6.4&6.3\\\hline
 A/California/62/2018 &H1N2& Jul  2019 &5.8&5.7&37&37&-0.2038&-0.0477&0.0986&5.3&5.9\\\hline
 A/Bangladesh/0994/2011 &H9N2& Feb  2014 &5.6&5.4&58&58&-0.0473&-0.4654&0.1484&3.8&3.6\\\hline
 A/Sichuan/06681/2021 &H5N6& Oct  2021 &5.3&6.3&46&46&-0.3443&-0.0600&0.1437&5.1&6.2\\\hline
 A/Vietnam/1203/2004 &H5N1& Nov  2011 &5.2&6.6&48&45&-0.1323&-0.0411&0.0738&5.6&5.8\\\hline
 A/Yunnan/14564/2015 &H5N6& Apr  2016 &5.0&6.6&46&46&-0.2187&-0.0415&0.0953&5.4&6.0\\\hline
 A/Astrakhan/3212/2020 &H5N8& Mar  2021 &4.6&5.2&95&92&-0.2366&-0.5451&0.3591&4.8&6.1\\\hline
 A/Netherlands/219/2003 &H7N7& Jun  2012 &4.6&5.8&1000&1000&-0.1658&-0.4596&0.2760&3.9&4.5\\\hline
 A/American  wigeon/South  Carolina/AH0195145/2021 &H5N1& Mar  2022 &4.4&5.1&48&45&-0.2355&-0.3135&0.2717&4.3&5.2\\\hline
 A/Jiangxi-Donghu/346/2013$^{\star\star}$ &H10N8& Feb  2014 &4.3&6.0&&&-0.2097&-0.2299&0.2196&4.2&4.8\\\hline
 A/gyrfalcon/Washington/ 41088/2014 &H5N8& Mar  2015 &4.2&4.6&95&92&-0.2387&-0.5438&0.3603&4.8&6.1\\\hline
 A/Northern  pintail /Washington/40964/2014 &H5N2& Mar  2015 &3.8&4.1&95&92&-0.2327&-0.5099&0.3445&4.6&5.8\\\hline
 A/canine/Illinois/12191/2015 &H3N2& Jun  2016 &3.7&3.7&1000&1000&-0.0179&-0.0374&0.0259&6.2&6.1\\\hline
 A/American  green-winged  teal /Washington/1957050/2014 &H5N1& Mar  2015 &3.6&4.1&48&45&-0.2352&-0.3067&0.2686&4.3&5.1\\\hline
 A/turkey/Indiana/1573-2/2016 &H7N8& Jul  2017 &3.4&3.9&1000&1000&-0.0438&-0.4165&0.1351&4.0&3.8\\\hline
 A/chicken/Tennessee/17-007431-3/2017 &H7N9& Oct  2017 &3.1&3.5&1000&1000&-0.0335&-0.5127&0.1310&3.8&3.6\\\hline
 A/chicken/Tennessee/17-007147-2/2017 &H7N9& Oct  2017 &2.8&3.5&1000&1000&-0.0839&-0.5127&0.2075&3.5&3.6\\\hline
 %A/duck/New  York/1996$^\star$ &H1N1& Nov  2011 &2.3&2.4&1000&1000&-1&-1&-1&-1&-1\\\hline
\end{tabular}
\flushleft

\fontsize{8}{8}\selectfont
$^\star$This table contains \enet scores for IRAT computed using current sequence data, thereby computing the current risk of these strains.  -1 indicates missing data, either from lack of human sequence data available for that virus sub-type (less than 30 strains) or missing IRAT sequence data (in the case of A/duck/New York/1996)
\end{table}
\else
\refstepcounter{table}\label{irattab_current}
\fi
% #############################################
% #############################################



%#############################################
%#############################################
\begin{table}\centering
\captionN{General linear model for evaluating effect of data diversity on \enet performance}\label{tabreg}\centering

\begin{tabular}{L{1.5in}|L{2in}}\hline
  Variable Name & Description \\\hline
  qnet\_complexity & Cumulative number of nodes in all predictors in the corresponding \enet \\\hline
  data\_diversity &  Number of clusters in set of input sequence where each sequence in a specific cluster is separated by at least $5$ mutations from sequences not in the cluster\\\hline
  ldistance\_WHO & Deviation of WHO predicted strain from the dominant strain\\\hline
\end{tabular}
\vskip 2em 

\mnp{6.5in}{
  \fontsize{8}{8}\selectfont
\verbatiminput{Figures/tabdata/model1.txt}
}
\vskip 2em


\mnp{6.5in}{
  \fontsize{8}{8}\selectfont
\verbatiminput{Figures/tabdata/model2.txt}
}
\end{table}
%#############################################
%#############################################





%#############################################
%#############################################
\begin{table}\centering
\captionN{General linear model evaluating \enet emergence risk predictions against IRAT estimates}\label{tabreg}\centering

\mnp{6.5in}{
  \fontsize{8}{8}\selectfont
\verbatiminput{Figures/tabdata/model_simple.txt}
}
\vskip 2em


\mnp{6.5in}{
  \fontsize{8}{8}\selectfont
\verbatiminput{Figures/tabdata/model_complex.txt}
}
\end{table}
%#############################################
%#############################################





% %#############################################
% %#############################################
% \begin{table}
%   \centering
% \captionN{Numbering Conversion to pdm09 and H3 Schemes}\label{tabnum}
  
%   \fontsize{5}{5}\selectfont

%   \mnp{1.5in}{\input{Figures/tabdata/col1.tex}}
%   \mnp{1.5in}{\input{Figures/tabdata/col2.tex}}
%   \mnp{1.5in}{\begin{tabular}{L{.2in}|L{.32in}|L{.15in}}
Query &H1N1pdm&H3\\\hline
157&140&143\\\hline
158&141&144\\\hline
159&142&145\\\hline
160&143&146\\\hline
161&144&147\\\hline
162&145&148\\\hline
163&146&149\\\hline
164&147&150\\\hline
165&148&151\\\hline
166&149&152\\\hline
167&150&153\\\hline
168&151&154\\\hline
169&152&155\\\hline
170&153&156\\\hline
171&154&157\\\hline
172&155&158\\\hline
-&-&-\\\hline
-&-&-\\\hline
-&-&-\\\hline
-&-&-\\\hline
173&156&159\\\hline
174&157&160\\\hline
175&158&161\\\hline
176&159&162\\\hline
177&160&163\\\hline
178&161&164\\\hline
179&162&165\\\hline
180&163&166\\\hline
181&164&167\\\hline
182&165&168\\\hline
183&166&169\\\hline
184&167&170\\\hline
-&-&-\\\hline
185&168&171\\\hline
186&169&172\\\hline
187&170&173\\\hline
-&-&-\\\hline
188&171&174\\\hline
189&172&175\\\hline
190&173&176\\\hline
191&174&177\\\hline
192&175&178\\\hline
193&176&179\\\hline
194&177&180\\\hline
195&178&181\\\hline
196&179&182\\\hline
197&180&183\\\hline
198&181&184\\\hline
199&182&185\\\hline
200&183&186\\\hline
201&184&187\\\hline
202&185&188\\\hline
203&186&189\\\hline
204&187&190\\\hline
205&188&191\\\hline
206&189&192\\\hline
207&190&193\\\hline
208&191&194\\\hline
209&192&195\\\hline
210&193&196\\\hline
211&194&197\\\hline
212&195&198\\\hline
213&196&199\\\hline
-&-&-\\\hline
214&197&200\\\hline
215&198&201\\\hline
216&199&202\\\hline
217&200&203\\\hline
218&201&204\\\hline
219&202&205\\\hline
220&203&206\\\hline
221&204&207\\\hline
222&205&208\\\hline
223&206&209\\\hline
224&207&210\\\hline
225&208&211\\\hline
226&209&212\\\hline
227&210&213\\\hline
228&211&214\\\hline
229&212&215\\\hline
230&213&216\\\hline
231&214&217\\\hline
232&215&218\\\hline
233&216&219\\\hline
234&217&220\\\hline
235&218&221\\\hline
236&219&222\\\hline
237&220&223\\\hline
-&-&-\\\hline
-&-&-\\\hline
\end{tabular}
}
%   \mnp{1.5in}{\begin{tabular}{L{.2in}|L{.32in}|L{.15in}}
Query &H1N1pdm&H3\\\hline
-&-&-\\\hline
-&-&-\\\hline
-&-&-\\\hline
238&221&224\\\hline
239&222&225\\\hline
240&223&226\\\hline
241&224&227\\\hline
242&225&228\\\hline
243&226&229\\\hline
244&227&230\\\hline
245&228&231\\\hline
246&229&232\\\hline
247&230&233\\\hline
248&231&234\\\hline
249&232&235\\\hline
250&233&236\\\hline
251&234&237\\\hline
252&235&238\\\hline
253&236&239\\\hline
254&237&240\\\hline
255&238&241\\\hline
256&239&242\\\hline
257&240&243\\\hline
258&241&244\\\hline
259&242&245\\\hline
260&243&246\\\hline
261&244&247\\\hline
262&245&248\\\hline
263&246&249\\\hline
264&247&250\\\hline
265&248&251\\\hline
266&249&252\\\hline
267&250&253\\\hline
268&251&254\\\hline
269&252&255\\\hline
270&253&256\\\hline
271&254&257\\\hline
272&255&258\\\hline
273&256&259\\\hline
274&257&260\\\hline
275&258&261\\\hline
276&259&262\\\hline
-&-&-\\\hline
-&-&-\\\hline
-&-&-\\\hline
-&-&-\\\hline
-&-&-\\\hline
-&-&-\\\hline
-&-&-\\\hline
-&-&-\\\hline
-&-&-\\\hline
277&260&-\\\hline
278&261&263\\\hline
279&262&264\\\hline
280&263&265\\\hline
281&264&266\\\hline
282&265&267\\\hline
283&266&268\\\hline
284&267&269\\\hline
285&268&270\\\hline
286&269&271\\\hline
287&270&272\\\hline
288&271&273\\\hline
289&272&274\\\hline
290&273&275\\\hline
291&274&276\\\hline
292&275&277\\\hline
293&276&278\\\hline
294&277&279\\\hline
295&278&280\\\hline
296&279&281\\\hline
297&280&282\\\hline
298&281&283\\\hline
299&282&284\\\hline
300&283&285\\\hline
-&-&-\\\hline
301&284&286\\\hline
302&285&287\\\hline
303&286&288\\\hline
304&287&289\\\hline
305&288&290\\\hline
306&289&291\\\hline
307&290&292\\\hline
308&291&293\\\hline
309&292&294\\\hline
310&293&295\\\hline
311&294&296\\\hline
-&-&-\\\hline
312&295&297\\\hline
313&296&298\\\hline
\end{tabular}
}

% \end{table}