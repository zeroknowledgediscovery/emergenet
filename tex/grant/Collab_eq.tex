\textbf{The University of Chicago} is a private non-profit institution located on the ethnically-diverse South Side of Chicago that has been a center of advanced learning and research since its inception in 1892. The University of Chicago is comprised of four graduate Divisions (Biological Sciences, Physical Sciences, Social Sciences, and Humanities), six professional schools (Chicago Booth School of Business, Divinity School, Harris School of Public Policy Studies, Law School, Pritzker School of Medicine, and School of Social Service Administration), the Graham School of General Studies, and the undergraduate College. The University has a unique history of organizing around research questions that cross disciplines rather than operating within  disciplinary boundaries. The extent to which this strategy reflects University of Chicago is illustrated by its numerous interdisciplinary Committees, Centers and Institutes (described below). The University of Chicago maintains its commitment to scholarship, teaching, and research through its more than 2100 faculty members and a student population of approximately 15,600 with nearly 2/3 engaged in advanced research and professional study. Through the years, 86 Nobel Laureates (8 are current faculty),  44 members of the National Academy of Sciences, 169 members of the American Academy of Arts and Sciences, and 14 recipients of the National Medal of Science have been associated with the University as students, teachers or research investigators. The University of Chicago is ranked among the world’s top universities by a number of criteria, including the amount of federal research funding received (despite a size much smaller than many of its academic peers). This spirit of discovery, innovation and public service provides a robust foundation for success.

\textbf{South Shore Senior Health Center (SSSC):} The SSSC is a 6800 sq. ft. university-owned geriatric facility ~5 miles south of DCAM, with doorstep free parking in the heart of Chicago’s South Side, specifically in the South Shore district. There are 13 patient exam rooms equipped with an examination bed, ophthalmoscope and otoscope, in addition to an on-site phlebotomist and capability for ECG. There is a large conference room for team meetings and support groups. The Memory Center team meets on Mondays at this site.

\textbf{Center for Care and Discovery:} Completed in 2013, the CCD is a ten-story adult hospital focused on cancer, advanced surgery, high  tech  imaging,  the  neurosciences  and  gastrointestinal  procedures. The building is 1,200,000 GSF with floor plates of 102,000 GSF. The CCD includes 240 private patient rooms, 28 operating rooms (21 initially); an imaging department with 3 CT’s, 2 MRI’s, 1 fluoroscopy room, 2 general radiology rooms, 7 interventional radiology rooms; and a gastroenterology procedures suite with 11 GI procedure rooms, 2 fluoroscopy rooms and 2 bronchoscopy rooms.  The CCD includes an inpatient kitchen, cafeteria, 7th floor sky lobby meeting rooms, ground floor retail space and clinical support services. Two floors are “shelled space” for future expansion of services. The building is centrally located between existing clinical facilities (DCAM, Comer) and new research facilities (KCBD, GCIS, Knapp). The facility is connected to both DCAM and Comer via above and below ground connections. There are procedure rooms, 2 fluoroscopy rooms and 2 bronchoscopy rooms. Neurology and neurosurgery inpatients as well as a Neuro- ICU are contiguously located on one floor of the CCD. 

% \textbf{Department of Neurology:} The Department of Neurology, chaired by Dr. Shyam Prabhakaran, is a nationally recognized leader in neuroscience research. The department consists of 28 primary faculty and 21staff members. The department has multiple funded investigators and over 3 million dollars in extramural research. Along with the university and the BSD, the Department of Neurology places primary emphasis on the training of the next generation of academic leaders in neurology with a special focus on the development of clinician scientists. There has been expansion of the residency from 18 to 21 (7 per year). The department with UCM has launched the Clinical Neurosciences Service Line in 2019, directed by Dr. Prabhakaran, to integrate clinical care in neurology and neurosurgery across the spectrum of neurologic disease including community health and prevention, hospital-based acute care, and post-acute rehabilitation and re-integration in the community. The Service Line partners with the hospital and university to bring together resources including bioinformatics, quality improvement, clinical trials and research, and tele-technologies to enhance patient experience and outcomes.

% \textbf{Office Space:} The Neurology offices are housed in the UCM campus and medical center. It has 3 available conference rooms, teleconference capabilities, video conferencing capabilities both informally via internet-available software, as well as formally with WebEx. Dr. Mastrianni and the Executive Administrator will have private offices, as will the Outreach Core leader, and nurse. The coordinator and other staff members have dedicated workspaces in the Neurology office suite. All offices include: phone, pager, computer, scanner, fax, high-speed internet access and administrative support. Standard software available includes: SPSS, EndNote, Microsoft Access, and Adobe Acrobat Professional. The PD’s 300 sq foot office is equipped with a locked door and ample file storage. The research staff members are also equipped with locking file storage. In addition, the suite is protected by key card restricted access.

% \textbf{Memory Center:} Roughly 650 new patients and 1500 return patients are evaluated each year at the Memory Center. This is expected to rise, as an additional Behavioral Neurologist is being recruited as of this writing. Recruitment of most subjects will come from the Memory Center clinics. Two clinic sites are utilized. 1) the Duchossois Center for Advanced Medicine (DCAM) on the University of Chicago Medicine campus, and; 2) the South Shore Senior Health Center (SSSC), a 6800 sq. ft. university-owned geriatric facility approximately 5 miles south of DCAM, with convenient doorstep parking in the heart of Chicago’s South Side. Within the SSSC, in addition to referrals from the general geriatricians, there are specialty clinics that act as a referral to the Memory Center, including the Successful Aging and Frailty Evaluation (SAFE). The catchment area for the SSSC includes a large segment of south Chicago and northwestern Indiana, which currently has limited access to geriatrics and dementia specialists. Roughly 80\% of patients with cognitive problems at the SSSC are African American. 