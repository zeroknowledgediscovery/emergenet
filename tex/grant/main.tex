\documentclass[onecolumn, compsoc,12pt]{IEEEtran}
\usepackage{enumitem}
\usepackage{etex}
\usepackage{amssymb,amsfonts,amsmath,amsthm}
\usepackage{graphicx}
\usepackage{booktabs}
\usepackage[usenames,x11names, dvipsnames, svgnames]{xcolor}
\usepackage{amsmath,amssymb}
\usepackage{dsfont}
\usepackage{amsfonts}
\usepackage{mathrsfs}
\usepackage{texshade}
\usepackage{hyperref}
\hypersetup{
  colorlinks=true,
  linkcolor=black,
  citecolor=blue,
  filecolor=black,
  urlcolor=DodgerBlue4,
  breaklinks=false,
  % linkbordercolor=red,% hyperlink borders will be red
  % pdfborderstyle={/S/U/W 1}% border style will be underline of width 1pt
}
\usepackage{array}
\usepackage{xr}
\usepackage{verbatim}
\usepackage{multirow}
\usepackage{longtable}
\usepackage{tikz-network}
\usepackage[T1,euler-digits]{eulervm}
\usepackage{times}
% \usepackage{pxfonts}
\usepackage{tikz}
\usepackage{pgfplots}
\usetikzlibrary{shapes,calc,shadows,fadings,arrows,decorations.pathreplacing,automata,positioning}
\usetikzlibrary{external}
\usetikzlibrary{decorations.text}
\usepgfplotslibrary{colorbrewer} 
\usepgfplotslibrary{statistics}

\tikzexternalize[prefix=./Figures/External/]% activate externalization!
\tikzexternaldisable
% \addtolength{\voffset}{.1in}  
\usepackage{geometry}
\geometry{letterpaper, left=.6in,right=.6in,top=.5in,bottom=0.7in}

%\addtolength{\textwidth}{-.1in}    
%\addtolength{\hoffset}{.05in}    
%\addtolength{\textheight}{.1in}    
%\addtolength{\footskip}{0in}    
\usepackage{rotating}
\definecolor{nodecol}{RGB}{240,240,220}
\definecolor{nodeedge}{RGB}{240,240,225}
\definecolor{edgecol}{RGB}{130,130,130}
\tikzset{%
  fshadow/.style={      preaction={
      fill=black,opacity=.3,
      path fading=circle with fuzzy edge 20 percent,
      transform canvas={xshift=1mm,yshift=-1mm}
    }} 
}
\usetikzlibrary{pgfplots.dateplot}
\usetikzlibrary{patterns}
\usetikzlibrary{decorations.markings}
\usepackage{fancyhdr}
\usepackage{mathtools}
\usepackage{datetime}
\usepackage{comment}
%% ## Equation Space Control---------------------------
\def\EQSP{3pt}
\newcommand{\mltlne}[2][\EQSP]{\begingroup\setlength\abovedisplayskip{#1}\setlength\belowdisplayskip{#1}\begin{equation}\begin{multlined} #2 \end{multlined}\end{equation}\endgroup\noindent}
\newcommand{\cgather}[2][\EQSP]{\begingroup\setlength\abovedisplayskip{#1}\setlength\belowdisplayskip{#1}\begin{gather} #2 \end{gather}\endgroup\noindent}
\newcommand{\cgathers}[2][\EQSP]{\begingroup\setlength\abovedisplayskip{#1}\setlength\belowdisplayskip{#1}\begin{gather*} #2 \end{gather*}\endgroup\noindent}
\newcommand{\calign}[2][\EQSP]{\begingroup\setlength\abovedisplayskip{#1}\setlength\belowdisplayskip{#1}\begin{align} #2 \end{align}\endgroup\noindent}
\newcommand{\caligns}[2][\EQSP]{\begingroup\setlength\abovedisplayskip{#1}\setlength\belowdisplayskip{#1}\begin{align*} #2 \end{align*}\endgroup\noindent}
\newcommand{\mnp}[2]{\begin{minipage}{#1}#2\end{minipage}} 
%% COLOR DEFS------------------------------------------
\newtheorem{thm}{Theorem}
\newtheorem{cor}{Corollary}
\newtheorem{lem}{Lemma}
\newtheorem{prop}{Proposition}
\newtheorem{defn}{Definition}
\newtheorem{exmpl}{Example}
\newtheorem{rem}{Remark}
\newtheorem{notn}{Notation}
%% ------------PROOF INCLUSION -----------------
\def\NOPROOF{Proof omitted.}
\newif\ifproof
\prooffalse % or \draftfalse
\newcommand{\Proof}[1]{
  \ifproof
  \begin{IEEEproof}
    #1\end{IEEEproof}
  \else
  \NOPROOF
  \fi
}
%% ------------ -----------------
\newcommand{\DETAILS}[1]{#1}
%% ------------ -----------------
% color commands------------------------
\newcommand{\etal}{\textit{et} \mspace{3mu} \textit{al.}}
% \renewcommand{\algorithmiccomment}[1]{$/** $ #1 $ **/$}
\newcommand{\vect}[1]{\textbf{\textit{#1}}}
\newcommand{\figfont}{\fontsize{8}{8}\selectfont\strut}
\newcommand{\hlt}{ \bf \sffamily \itshape\color[rgb]{.1,.2,.45}}
\newcommand{\pitilde}{\widetilde{\pi}}
\newcommand{\Pitilde}{\widetilde{\Pi}}
\newcommand{\bvec}{\vartheta}
\newcommand{\algo}{\textrm{\bf\texttt{GenESeSS}}\xspace}
\newcommand{\xalgo}{\textrm{\bf\texttt{xGenESeSS}}\xspace}
\newcommand{\FNTST}{\bf }
\newcommand{\FNTED}{\color{darkgray} \scriptsize $\phantom{.}$}
\renewcommand{\baselinestretch}{.9}
\newcommand{\sync}{\otimes}
\newcommand{\psync}{\hspace{3pt}\overrightarrow{\hspace{-3pt}\sync}}
% \newcommand{\psync}{\raisebox{-4pt}{\begin{tikzpicture}\node[anchor=south] (A) {$\sync$};
%   \draw [->,>=stealth] ([yshift=-2pt, xshift=2pt]A.north west) -- ([yshift=-2pt]A.north east); %\end{tikzpicture}}}
\newcommand{\base}[1]{\llbracket #1 \rrbracket}
\newcommand{\nst}{\textrm{\sffamily\textsc{Numstates}}}
\newcommand{\HA}{\boldsymbol{\mathds{H}}}
\newcommand{\eqp}{ \vartheta }
\newcommand{\entropy}[1]{\boldsymbol{h}\left ( #1 \right )}
\newcommand{\norm}[1]{\left\lVert #1 \right\rVert}%
\newcommand{\abs}[1]{\left\lvert #1 \right\rvert}%
\newcommand{\absB}[1]{\big\lvert #1 \big\rvert}%
% #############################################################
% #############################################################
% PREAMBLE ####################################################
% #############################################################
% #############################################################
% \usepackage{pnastwoF}      
\DeclareMathOperator*{\argmax}{argmax}
\DeclareMathOperator*{\argmin}{arg\,min}
\DeclareMathOperator*{\expect}{\mathbf{E}}
\DeclareMathOperator*{\var}{\mathbf{Var}}

\newcommand{\ND}{ \mathcal{N}  }
\usepackage[linesnumbered,ruled,vlined,noend]{algorithm2e}
\newcommand{\captionN}[1]{\caption{\color{darkgray} \sffamily \fontsize{9}{10}\selectfont #1  }}
\newcommand{\btl}{\ \textbf{\small\sffamily bits/letter}}
%\usepackage{txfonts}
%%% \usepackage{ccfonts}
%%% save defaults
%\renewcommand{\rmdefault}{phv} % Arial
%\renewcommand{\sfdefault}{phv} % Arial
%\edef\keptrmdefault{\rmdefault}
%\edef\keptsfdefault{\sfdefault}
%\edef\keptttdefault{\ttdefault}

% \usepackage{kerkis}
%\usepackage[OT1]{fontenc}
%\usepackage{concmath}
% \usepackage[T1]{eulervm} 
% \usepackage[OT1]{fontenc}
%%% restore defaults
%\edef\rmdefault{\keptrmdefault}
%\edef\sfdefault{\keptsfdefault}
%\edef\ttdefault{\keptttdefault}
\tikzexternalenable
% ##########################################################
\tikzfading[name=fade out,
inner color=transparent!0,
outer color=transparent!100]
% ###################################
\newcommand{\xtitaut}[2]{
  \noindent\mnp{\textwidth}{
    \mnp{\textwidth}{\raggedright\Huge \bf \sffamily #1}

    \vskip 1em

    {\bf \sffamily #2}
  }
  \vskip 2em
}
% ###################################
% ###################################
\tikzset{wiggle/.style={decorate, decoration={random steps, amplitude=10pt}}}
\usetikzlibrary{decorations.pathmorphing}
\pgfdeclaredecoration{Snake}{initial}
{
  \state{initial}[switch if less than=+.625\pgfdecorationsegmentlength to final,
  width=+.3125\pgfdecorationsegmentlength,
  next state=down]{
    \pgfpathmoveto{\pgfqpoint{0pt}{\pgfdecorationsegmentamplitude}}
  }
  \state{down}[switch if less than=+.8125\pgfdecorationsegmentlength to end down,
  width=+.5\pgfdecorationsegmentlength,
  next state=up]{
    \pgfpathcosine{\pgfqpoint{.25\pgfdecorationsegmentlength}{-1\pgfdecorationsegmentamplitude}}
    \pgfpathsine{\pgfqpoint{.25\pgfdecorationsegmentlength}{-1\pgfdecorationsegmentamplitude}}
  }
  \state{up}[switch if less than=+.8125\pgfdecorationsegmentlength to end up,
  width=+.5\pgfdecorationsegmentlength,
  next state=down]{
    \pgfpathcosine{\pgfqpoint{.25\pgfdecorationsegmentlength}{\pgfdecorationsegmentamplitude}}
    \pgfpathsine{\pgfqpoint{.25\pgfdecorationsegmentlength}{\pgfdecorationsegmentamplitude}}
  }
  \state{end down}[width=+.3125\pgfdecorationsegmentlength,
  next state=final]{
    \pgfpathcosine{\pgfqpoint{.15625\pgfdecorationsegmentlength}{-.5\pgfdecorationsegmentamplitude}}
    \pgfpathsine{\pgfqpoint{.15625\pgfdecorationsegmentlength}{-.5\pgfdecorationsegmentamplitude}}
  }
  \state{end up}[width=+.3125\pgfdecorationsegmentlength,
  next state=final]{
    \pgfpathcosine{\pgfqpoint{.15625\pgfdecorationsegmentlength}{.5\pgfdecorationsegmentamplitude}}
    \pgfpathsine{\pgfqpoint{.15625\pgfdecorationsegmentlength}{.5\pgfdecorationsegmentamplitude}}
  }
  \state{final}{\pgfpathlineto{\pgfpointdecoratedpathlast}}
}
% ###################################
% ###################################
\newcolumntype{L}[1]{>{\rule{0pt}{2ex}\raggedright\let\newline\\\arraybackslash\hspace{0pt}}m{#1}}
\newcolumntype{C}[1]{>{\rule{0pt}{2ex}\centering\let\newline\\\arraybackslash\hspace{0pt}}m{#1}}
\newcolumntype{R}[1]{>{\rule{0pt}{2ex}\raggedleft\let\newline\\\arraybackslash\hspace{0pt}}m{#1}}



% ################################################
% ################################################
% ################################################
% ################################################
\def\DISCLOSURE#1{\def\disclosure{#1}}
\DISCLOSURE{\raisebox{15pt}{$\phantom{XxxX}$This sheet contains proprietary information 
    not to be released to third parties except for the explicit purpose of evaluation.}
}
% ####################################
\newcommand{\set}[1]{\left\{ #1 \right\}}
\newcommand{\paren}[1]{\left( #1 \right)}
\newcommand{\bracket}[1]{\left[ #1 \right]}
% \newcommand{\norm}[1]{\left\Vert #1 \right\Vert}
\newcommand{\nrm}[1]{\left\llbracket{#1}\right\rrbracket}
\newcommand{\parenBar}[2]{\paren{#1\,{\left\Vert\,#2\right.}}}
\newcommand{\parenBarl}[2]{\paren{\left.#1\,\right\Vert\,#2}}
\newcommand{\ie}{$i.e.$\xspace}
\newcommand{\addcitation}{\textcolor{black!50!red}{\textbf{ADD CITATION}}}
\newcommand{\subtochange}[1]{{\color{black!50!green}{#1}}}
\newcommand{\tobecompleted}{{\color{black!50!red}TO BE COMPLETED.}}


\newcommand{\pIn}{\mathscr{P}_{\textrm{in}}}
\newcommand{\pOut}{\mathscr{P}_{\textrm{out}}}
\newcommand{\aIn}[1][\Sigma]{#1_{\textrm{in}}}
\newcommand{\aOut}[1][\Sigma]{#1_{\textrm{out}}}
\newcommand{\xin}[1]{#1_{\textrm{in}}}
\newcommand{\xout}[1]{#1_{\textrm{out}}}

\newcommand{\R}{\mathbb{R}} % Set of real numbers
\newcommand{\F}[1][]{\mathcal{F}_{#1}}
\newcommand{\SR}{\mathcal{S}} % Semiring of sets
\newcommand{\RR}{\mathcal{R}} % Ring of sets
\newcommand{\N}{\mathbb{N}} % Set of natural numbers (0 included)


\newcommand{\Pp}[1][n]{\mathscr{P}^+_{#1}}
\renewcommand{\entropy}[1]{\boldsymbol{h}\left ( #1 \right )}



\makeatletter
\pgfdeclarepatternformonly[\hatchdistance,\hatchthickness]{flexible hatch}
{\pgfqpoint{0pt}{0pt}}
{\pgfqpoint{\hatchdistance}{\hatchdistance}}
{\pgfpoint{\hatchdistance-1pt}{\hatchdistance-1pt}}%
{
  \pgfsetcolor{\tikz@pattern@color}
  \pgfsetlinewidth{\hatchthickness}
  \pgfpathmoveto{\pgfqpoint{0pt}{0pt}}
  \pgfpathlineto{\pgfqpoint{\hatchdistance}{\hatchdistance}}
  \pgfusepath{stroke}
}
\makeatother

\pgfdeclarepatternformonly{north east lines wide}%
{\pgfqpoint{-1pt}{-1pt}}%
{\pgfqpoint{10pt}{10pt}}%
{\pgfqpoint{9pt}{9pt}}%
{
  \pgfsetlinewidth{0.7pt}
  \pgfpathmoveto{\pgfqpoint{0pt}{0pt}}
  \pgfpathlineto{\pgfqpoint{9.1pt}{9.1pt}}
  \pgfusepath{stroke}
}

\pgfdeclarepatternformonly{north west lines wide}%
{\pgfqpoint{-1pt}{-1pt}}%
{\pgfqpoint{10pt}{10pt}}%
{\pgfqpoint{9pt}{9pt}}%
{
  \pgfsetlinewidth{0.7pt}
  \pgfpathmoveto{\pgfqpoint{0pt}{9pt}}
  \pgfpathlineto{\pgfqpoint{9.1pt}{-0.1pt}}
  \pgfusepath{stroke}
}
\makeatletter

\pgfdeclarepatternformonly[\hatchdistance,\hatchthickness]{flexible hatchB}
{\pgfqpoint{0pt}{\hatchdistance}}
{\pgfqpoint{\hatchdistance}{0pt}}
{\pgfpoint{1pt}{\hatchdistance-1pt}}%
{
  \pgfsetcolor{\tikz@pattern@color}
  \pgfsetlinewidth{\hatchthickness}
  \pgfpathmoveto{\pgfqpoint{0pt}{\hatchdistance}}
  \pgfpathlineto{\pgfqpoint{\hatchdistance}{0pt}}
  \pgfusepath{stroke}
}    \makeatother


\def\TPR{\textrm{TPR}\xspace}
\def\TNR{\textrm{TNR}\xspace}
\def\FPR{\textrm{FPR}\xspace}
\def\PPV{\textrm{PPV}\xspace}

\usetikzlibrary{arrows.meta}
\usetikzlibrary{decorations.pathreplacing,shapes.misc}
\usepgfplotslibrary{fillbetween}
%usepackage{tikz-network}
\usetikzlibrary{shapes.geometric}
\usetikzlibrary{math}
\usepgfplotslibrary{colorbrewer} 

\usepackage{textcomp}
\usepackage{colortbl}
\usepackage{array}
\usepackage{courier} 
\usepackage{wrapfig}
\usepackage{pifont}
\usetikzlibrary{chains,backgrounds}
\usetikzlibrary{intersections}
\usetikzlibrary{pgfplots.groupplots}
\usepgfplotslibrary{fillbetween} 
\usetikzlibrary{arrows.meta}
\usepackage{pgfplotstable}
\usepackage[super,compress,sort,comma]{natbib}
%\usepackage{natbib}
\usepackage{setspace}
\usetikzlibrary{math}
\usetikzlibrary{matrix}
\usepackage{xstring}
\usepackage{xspace}
\usepackage{flushend}
\makeatletter
\renewcommand\section{\@startsection {section}{1}{\z@}%
  {-2ex \@plus -1ex \@minus -.2ex}%
  {1ex \@plus.1ex}%
  {\Large\bfseries\scshape}}
\renewcommand\subsection{\@startsection {subsection}{1}{\z@}%
  {-2ex \@plus -.25ex \@minus -.2ex}%
  {0.1ex \@plus.0ex}%
  {\fontsize{11}{10}\selectfont\bfseries\sffamily\color{black}}}
\renewcommand\subsubsection{\@startsection {subsubsection}{1}{\z@}%
  {0ex \@plus -.5ex \@minus -.2ex}%
  {0.0ex \@plus.5ex}%
  {\bfseries\itshape\sffamily\color{darkgray}}}
\renewcommand\paragraph{\@startsection {paragraph}{1}{\z@}%
  {-.2ex \@plus -.5ex \@minus -.2ex}%
  {0.0ex \@plus.5ex}%
  {\fontsize{9}{9}\selectfont\itshape\sffamily\color{darkgray}}}
       
%\renewcommand{\thesubsection}{\thesection.\arabic{subsection}}
\renewcommand{\thesubsectiondis}{\arabic{subsection}.}
\renewcommand{\thesectiondis}{\arabic{section}.}
\renewcommand{\thesection}{\arabic{section}}

\renewcommand{\thetable}{\arabic{table}}

\makeatother
\makeatletter
\pgfdeclareradialshading[tikz@ball]{ball}{\pgfqpoint{-10bp}{10bp}}{%
  color(0bp)=(tikz@ball!30!white);
  color(9bp)=(tikz@ball!75!white);
  color(18bp)=(tikz@ball!90!black);
  color(25bp)=(tikz@ball!70!black);
  color(50bp)=(black)}
\makeatother
%\newcommand{\tball}[1][CadetBlue4]{${\color{#1}\Large\boldsymbol{\blacksquare}}$}
\renewcommand{\baselinestretch}{1}
%\renewcommand{\captionN}[1]{\caption{\color{CadetBlue4!50!black} \sffamily \fontsize{9}{10}\selectfont #1  }}
\tikzexternaldisable 
\parskip=6pt
\parindent=0pt
%\newcommand{\Mark}[1]{\textsuperscript{#1}}
\pagestyle{fancy}

\newcounter{Dcounter}
\setcounter{Dcounter}{1}
\newcommand{\DQS}[1]{\marginpar{\tikzexternaldisable \tikz{\node[rounded corners=5pt,draw=none,thick,fill=black!10,font=\sffamily\fontsize{7}{8}\selectfont] {\mnp{.45in} {\color{Red3}\raggedright  \#\theDcounter.~#1}}; }}\stepcounter{Dcounter}\xspace}

\newcommand{\qn}[1][i]{\Phi_{#1}}
\newcommand{\D}[1][i]{\mathscr{D}\left ( {\Sigma_#1} \right ) }
\newcommand{\Dx}{\mathscr{D}}
\def\J{\mathds{J}}
\def\M{\omega}
\def\N{\mathds{N}}
\newcommand{\cp}[1][P]{\langle #1 \rangle}
\newcommand{\mem}[1]{\M_{#1}}


\makeatletter
\newcommand\transformxdimension[1]{
    \pgfmathparse{((#1/\pgfplots@x@veclength)+\pgfplots@data@scale@trafo@SHIFT@x)/10^\pgfplots@data@scale@trafo@EXPONENT@x}
}
\newcommand\transformydimension[1]{
    \pgfmathparse{((#1/\pgfplots@y@veclength)+\pgfplots@data@scale@trafo@SHIFT@y)/10^\pgfplots@data@scale@trafo@EXPONENT@y}
}
\makeatother

\parskip=6pt
\parindent=0pt


\pgfplotsset{
    discard if/.style 2 args={
        x filter/.code={
            \edef\tempa{\thisrow{#1}}
            \edef\tempb{#2}
            \ifx\tempa\tempb
                \def\pgfmathresult{inf}
            \fi
        }
    },
    discard if not/.style 2 args={
        x filter/.code={
            \edef\tempa{\thisrow{#1}}
            \edef\tempb{#2}
            \ifx\tempa\tempb
            \else
                \def\pgfmathresult{inf}
            \fi
        }
    }
  }

  %\newcommand{\HLT}[2][Red1]{{\color{#1}#2}}

 % \def\commatonone{\expandafter\zappointzerozero
%    \romannumeral`\^^@}
%\def\zappointzerozero#1.00{\zapcomma#1,!}
%\def\zapcomma#1,#2{#1\ifx!#2\else#2\expandafter\zapcomma\fi}
\def\commatononei#1,{#1}
\def\commatononej#1,#2,{#1#2}
\def\commatonone#1{\expandafter\commatononei#1}
\def\commatononeT#1{\expandafter\commatononej#1}
\newcommand{\Sum} [2] {#1 + #2 = \the\numexpr #1 + #2 \relax \\}


\usepackage{sistyle}
\SIthousandsep{,}

\makeatletter
\newcommand{\limitpages}[1]{
  \gdef\maxpages{#1}%
  \ifx\latex@outputpage\@undefined\relax%
  \global\let\latex@outputpage\@outputpage%
  \fi%
  \gdef\@outputpage{%
    \ifnum\value{page}>\maxpages\relax%
    % Do not output the page
    \else%
    \latex@outputpage%
    \fi%
  }%
}
\makeatother
\newcommand{\note}[1]{{ \itshape \footnotesize \color{Red1}$\medbullet$~ #1}}









\renewcommand{\thesectiondis}{\arabic{section}.}
\renewcommand{\thesubsectiondis}{\Alph{subsection}.}

\makeatletter
\renewcommand\section{\@startsection {section}{1}{\z@}%
  {-1pt \@plus -30ex \@minus 20ex}%
  {.1pt}%
  {\large\bfseries\scshape}}
\renewcommand\subsection{\@startsection {subsection}{2}{\z@}%
  {0ex \@plus -1.75ex \@minus -1.2ex}%
  {0ex \@plus.0ex}%
  {\fontsize{11}{11}\selectfont\bfseries\sffamily\color{black}}}
\renewcommand\subsubsection{\@startsection {section}{1}{\z@}%
  {-.1ex \@plus -.5ex \@minus -.2ex}%
  {0.0ex \@plus.5ex}%
  {\bfseries\sffamily\color{Red4}}}
\renewcommand\paragraph{\@startsection {section}{1}{\z@}%
  {-.1ex \@plus -.5ex \@minus -.2ex}%
  {0.0ex \@plus.5ex}%
  {\fontsize{11}{10}\selectfont\bfseries\itshape\sffamily\color{black}}}
\makeatother


\makeatletter
\pgfdeclareradialshading[tikz@ball]{ball}{\pgfqpoint{-10bp}{10bp}}{%
  color(0bp)=(tikz@ball!30!white);
  color(9bp)=(tikz@ball!75!white);
  color(18bp)=(tikz@ball!90!black);
  color(25bp)=(tikz@ball!70!black);
  color(50bp)=(black)}
\makeatother
\newcommand{\tball}{${\color{CadetBlue3}\Large\boldsymbol{\blacksquare}}$}
\renewcommand{\baselinestretch}{.87}
\newcommand{\VSP}{\vspace{-2pt}}
\renewcommand{\captionN}[1]{\caption{\color{black} \sffamily \fontsize{9}{10}\selectfont #1  }}




\newcommand*{\doi}[1]{\href{http://dx.doi.org/#1}{doi: #1}}
\renewcommand{\IEEEbibitemsep}{20pt plus 2pt}
\makeatletter
\IEEEtriggercmd{\reset@font\normalfont\fontsize{11}{14}\selectfont}
\makeatother
\IEEEtriggeratref{1}
\newlength{\bibitemsep}\setlength{\bibitemsep}{.2\baselineskip plus .05\baselineskip minus .05\baselineskip}
\newlength{\bibparskip}\setlength{\bibparskip}{0pt}
\let\oldthebibliography\thebibliography
\renewcommand\thebibliography[1]{%
  \oldthebibliography{#1}%
  \setlength{\parskip}{\bibitemsep}%
  \setlength{\itemsep}{\bibparskip}%
}
\setlength{\bibitemsep}{.3\baselineskip plus .05\baselineskip minus .05\baselineskip}
\def\V{\mathds{V}}
\def\Appendix{Appendix}
%###################################

\newif\iftikzX
\tikzXtrue
\tikzXfalse
%--------------
\def\jobnameX{zero}
%--------------
\newif\ifFIGS
\FIGSfalse 
\FIGStrue
%--------------
\newif\ifdraftQ
\draftQtrue
%\draftQfalse
%--------------
%###################################
\def\TITLE{BioNORAD: Fast Scalable Pandemic Risk Assessment of \infl Strains Circulating In Non-human Hosts}
%\def\TITLE{\LARGE A Biologically Meaningful Sequence Metric\\For Analyzing Evolutionary Changes\\In Novel Pathogens}
%\def\TITLE{Learning  Mutational Patterns At Scale For\\Analysis Of Sequence Divergence\\In Novel Pathogens}
%\def\TITLE{Learning Mutational Patterns at Scale to Analyze Sequence Divergence in Novel Pathogens}

\def\authore{Kevin Wu}
\def\authora{ Jin Li}
\def\authorc{Aaron Esser-Kahn}
\def\authord{Ishanu Chattopadhyay}

\def\addressa{Department of Medicine, University of Chicago, IL, USA}
\def\addressb{Committee on Genetics, Genomics \& Systems Biology, University of Chicago, IL, USA}
\def\addressc{Committee on Quantitative Methods in Social, Behavioral, and Health Sciences, University of Chicago, IL, USA}
\def\addressd{Pritzker School of Molecular Engineering, University of Chicago, Chicago, IL, USA}
\def\addresse{Committee on Immunology, University of Chicago, Chicago, IL, USA}
\newif\ifdraftQ
\draftQtrue
\draftQfalse


%###################################

\title{\LARGE \TITLE}
% \author{\sffamily  \fontsize{10}{12}\selectfont   \authore$^{1}$,\authora$^{1}$,  \authorc$^{2,3}$, and \authord$^{1,4,5\bigstar}$\\                                                                
% \vspace{10pt}                                                                   

% \sffamily  \fontsize{10}{12}\selectfont                                         
% $^{1}$\addressa\\   
% $^{2}$\addressd\\
% $^{3}$\addresse\\
% $^{5}$\addressc                                                                 
% \vskip 1em                                                                      
% $^\bigstar$To whom correspondence should be addressed: e-mail: \texttt{ishanu@uchicago.edu}.
%}


\def\hcov{SARS-CoV-2\xspace}
\def\RATG13{RaTG13\xspace}
\def\Appendix{Appendix}
\def\qnet{Enet\xspace}
\def\enet{Emergenet\xspace}
\def\erisk{E-risk\xspace}
\def\qdist{E-distance\xspace}
\def\cov{COVID-19\xspace}
\def\infl{Influenza A\xspace}
%\def\infl{IAV\xspace}


\def\E{\mathcal{E}}
\def\dst{x_\star^{t+\delta}}
\def\dsta{x^{t+\delta}}
 
\usepackage{pgfgantt}
\usepackage{textcomp}
\usepackage{colortbl}
\usepackage{subfigure}
\usepackage{array}
\usepackage{courier}
\usepackage{setspace} 
\usepackage{wrapfig} 
\usepackage{calligra}
%\usepackage{ulem}
\usepackage{multirow}



\usetikzlibrary{chains,backgrounds}
\usetikzlibrary{intersections}
\usepackage{xstring}
\usepackage{wasysym}
\usepackage[misc]{ifsym}
\tikzexternaldisable 
\parskip=4pt
\parindent=0pt
\lhead{}
\pagestyle{fancy}
\lfoot{\thepage}
%\pagestyle{empty}
\def\COLA{black}
% ###################################
\cfoot{\bf\sffamily \scriptsize \color{Maroon!50} I. Chattopadhyay, Department of Medicine, University of Chicago}
\cfoot{}
\rhead{}
\newcommand{\partxt}{\bf\sffamily\itshape}
% ############################################################
\newif\iftikzX
\tikzXtrue
\tikzXfalse

\newcommand\guline{\bgroup\markoverwith
  {\textcolor{black!30}{\rule[-0.45ex]{2pt}{0.4pt}}}\ULon}
\newcommand\hilit[1]{\textcolor{Red1}{#1}}
\newcommand\hilitx[1]{\guline{#1}}
% ############################################################
\addtolength{\voffset}{.1in}
\addtolength{\textwidth}{-.085in}
\addtolength{\hoffset}{.0425in}
\def\PROG{Mallinckrodt\xspace}
\def\ZERO{ACoR\xspace}
\def\COLWA{\XCOLA!40}
\def\COLWB{\XCOLD!20}
\def\COLWC{\XCOLA!40}
\def\COLWD{\XCOLD!20}
\def\COLWE{\XCOLA!40}
\def\COLWF{\XCOLD!20}
% ############################################################
\def\treatment{positive\xspace}
\def\TITLE{A pharmacogenomic hypothesis for reducing the population risk of autism via potential off-label non-contra-indicated use of a common SSRI agent in pregnant women}
\def\TITLE{A pharmacogenomic hypothesis for reducing   risk of autism}
\def\TOTALCOSTSAMPLES{$\$1875 \times 7 = \$13,125$}
\def\PINAME{Ishanu Chattopadhyay}
\def\PIINST{University of Chicago}
\def\PIEMAIL{\url{ishanu@uchicago.edu}}

\def\SIMONDATA{SSC proband \& matched designated sibling ($672$ samples)}

\def\acor{ACoR\xspace}

\newcommand{\HDR}{
\begin{tabular}{|L{.3\textwidth} | L{.32\textwidth} | L{.3\textwidth} | }\hline
  Principal investigator Name: \bf \PINAME & Principal Investigator Institution: \bf \PIINST & Principal Investigator  email: \PIEMAIL \\\hline
  \multicolumn{2}{|C{.62\textwidth}|}{\hspace{-15pt} \mnp{.63\textwidth}{\vskip .45em Project Title\\ \bf \TITLE \\\vspace{-8pt} }} &  {\mnp{.3\textwidth}{\vskip .3em Project Type: \\ \bf Pilot}} \\\hline
  \multicolumn{3}{|C{.92\textwidth}|}{ \hspace{-40pt} \mnp{.93\textwidth}{\vskip .6em Data requested from Simons Collection:  \SIMONDATA \\ \vspace{-8pt}}} \\\hline
  \multicolumn{2}{|C{.62\textwidth}|}{\hspace{-15pt} \mnp{.63\textwidth}{Total Estimated Cost for Samples (Price List)}}& \TOTALCOSTSAMPLES\\\hline
\end{tabular}
\vskip .5em
}


\begin{document} 

\limitpages{5}
%\HDR

\section*{Project Narrative}
% – Rationale: Clearly articulate the scientific rationale for the proposed research
% project. Cite relevant literature. The presentation of preliminary and/or published
% data is allowed but not required.
% – Hypothesis: State concisely the new insights, paradigms, technologies, or
% applications that address one of the FY23 PRMRP Topic Areas and one of the
% associated FY23 PRMRP Strategic Goals.
% – Specific Aims: Concisely explain the project’s specific aims and the objective(s) to
% be reached. These aims should agree with the primary aims and associated tasks
% described in the Statement of Work (SOW). If the proposed work is part of a larger
% study, present only aims that this DOD award would fund.
% – Research Strategy and Feasibility: Describe the experimental design, methods, and
% analyses, including appropriate controls, in sufficient detail for scientific evaluation.
% Address potential problem areas and present alternative methods or approaches. If
% cell lines or animals are to be used, justify why the proposed cell line(s) or animal
% model(s) were chosen. Describe how the proposed project will be completed within
% the proposed performance period.

\textbf{Rationale:}
Animal influenza  viruses emerging into humans % are suspected to
have triggered devastating  pandemics in the past. Yet, our ability to evaluate the pandemic potential of individual strains that do not yet circulate in humans, remains limited. Here we propose to develop an experimentally  validated platform called the \enet (Enet), to predict in near-real-time where and when  new variants of concern would emerge,  using only observed  sequences  of key viral proteins, procured in ongoing global surveillance of \infl viruses. We bring together new machine learning algorithms customized to the problem at hand, key insights from information theory, evolutionary theory, epidemiology and precise statistical  uncertainty quantification to develop a rigorous framework, to  track evolutionary trajectories of pathogens through a complex, poorly characterized, and dynamically changing fitness landscape. Our deliverable is best described as the foundations for creating a platform akin to bio-NORAD,  \textit{identifying when and where an imminent zoonotic emergence event is likely, and if such novel strains are likely to achieve human-to-human transmission capability.}

  
Influenza viruses constantly evolve~\cite{dos2016influenza}, sufficiently altering surface protein structures to evade the prevailing host immunity, and cause the recurring seasonal  epidemic. These periodic  infection peaks claim a quarter to half a million lives~\cite{huddleston2020integrating} globally. Additionally, \infl, partly on account of its segmented genome and its wide prevalence in animal hosts, can easily incorporate genes from multiple strains and (re)emerge as novel human pathogens~\cite{reid2003origin}, thus harboring  a high pandemic potential. Strains spilling over into humans from animal reservoirs is thought to have triggered  pandemics  at least four times (1918 Spanish flu/H1N1, 1957 Asian flu/H2N2, 1968 Hong Kong flu/H3N2, 2009 swine flu/H1N1) in the past 100 years~\cite{shao2017evolution}. One  approach to mitigating such risk is to identify  animal strains  that do not yet circulate in humans, but is likely to spill-over and quickly achieve human-to-human (HH) transmission capability. While global surveillance efforts collect wild specimens from diverse hosts and geo-locations annually, our  ability to objectively, reliably and scalably  risk-rank individual strains remains limited~\cite{wille2021accurately}. The Center for Disease Control's (CDC) current solution to this problem is the Influenza Risk Assessment Tool (IRAT)~\cite{Influenz24:online}, which relies on time-consuming proteomics and transmission assays and potentially subjective evaluations by subject matter experts, taking  weeks to months to compile for each strain of concern. With tens of   thousands of strains being sequenced annually, this results in  a scalability bottleneck. 

Here we plan to develop a platform powered by novel pattern discovery and recognition algorithms to automatically parse out emergent evolutionary constraints operating on \infl viruses in the wild, to provide a less-heuristic theory-backed scalable solution to emergence prediction. % Our approach is centred around numerically estimating the probability $Pr(x \rightarrow y)$ of a strain $x$ spontaneously giving rise to  $y$.
We plan to show   that this capability enables preempting  strains which are expected to be in future human circulation, and  approximate IRAT scores of non-human strains without  experimental assays or SME scoring, in second as opposed to weeks or months. Our approach automatically takes into account the time-sensitive variations in selection pressures as the background strain circulation changes over time, and will potentially be able to rank-order strains adapatively. Additionally, we plan to validate our ability to predict future variations of viral proteins by showing that predicted variants of HA and NA fold correctly, and are functional, binding to the relevant human receptors in in-vivo laboratory experiments. Thus, bringing together rigorous data-driven modeling, and validation via tools from reverse genetics we plan to deliver an actionable and deployable platform that optimally exploits the current biosurvellance capacity.

The BioNORAD platform will enable proactive and actionable global surveillance for emerging pandemic threats from \infl. This importance of the ability to preempt pandemic risk to the national interest of the United States cannot be overstated, especially in the context of protecting  DoD assest and personnel deployed in potentially high risk centers of emergence. Additionally, the BioNORAD will enable preemptive action including the inoculation of  animal reservoirs before the first human infection, potentially eliminating the pandemic before it has a chance to  trigger.

\textbf{Hypotheses:} %State concisely the new insights, paradigms, technologies, or
% applications that address one of the FY23 PRMRP Topic Areas and one of the
% associated FY23 PRMRP Strategic Goals.

{\bf \itshape FY23 PRMRP Portfolio Category: Infectious Diseases $\vert$ FY23 PRMRP Topic: proteomics $\vert$  FY23 PRMRP Strategic Goal: Epidemiology: Identify strategies for surveillance or develop modeling tools and/or biomarkers to predict outbreaks or epidemics}

1) Learning patterns of cross-dependency between mutations and genomic change reveals enough of the underlying rules of organization to meaningfully and actionably constrain the evolutionary trajectories of emerging pathogens, and in our context, that of \infl viruses in the wild.

2) Such patterns can be learned from biosurveillance data that is being collected now globally to ultimately develop a next-generation pro-active surveillance platform that acts as an early warning system for pandemic threats, and serves a similar function to the strategic goal of NORAD in the context of defending our airspace from adversarial intrusion.

\textbf{Specific Aims:} % – Specific Aims: Concisely explain the project’s specific aims and the objective(s) to
% be reached. These aims should agree with the primary aims and associated tasks
% described in the Statement of Work (SOW). If the proposed work is part of a larger
% study, present only aims that this DOD award would fund.



\textbf{Research Strategy and Feasibility:} % – Research Strategy and Feasibility: Describe the experimental design, methods, and
% analyses, including appropriate controls, in sufficient detail for scientific evaluation.
% Address potential problem areas and present alternative methods or approaches. If
% cell lines or animals are to be used, justify why the proposed cell line(s) or animal
% model(s) were chosen. Describe how the proposed project will be completed within
% the proposed performance period.

\textbf{Innovation:} %Innovation: Describe how the proposed research is innovative, including how it will
%provide new insights, paradigms, technologies, or applications to the research field
%and/or patient care. Investigating the next logical step of an existing line of research
%or providing an incremental advance on published data is not considered innovative.
While numerous tools exist for ad hoc quantification of genomic similarity~\cite{posada1998modeltest,goldberger2005genomic,huelsenbeck1997phylogeny,neher2014predicting,VanderMeer2010,Smith2009}, higher similarity between strains in  these frameworks is not sufficient to imply a high likelihood of a jump. To the best of our knowledge, the \enet algorithm is  the first of its kind to learn an appropriate biologically meaningful comparison metric from data, without assuming any model of DNA or amino acid substitution, or a genealogical tree a priori. While the effect of the environment and selection cannot be inferred from a single sequence, an entire database of observed strains, processed through the right lens, can parse out useful predictive models of these complex interactions. Our results are  aligned with recent studies demonstrating effective  predictability of  future mutations  for different organisms~\cite{mollentze2021identifying,maher2021predicting}.

\clearpage
\mbox{}
\clearpage
\mbox{}
\clearpage
\mbox{}
\clearpage
\mbox{}
\clearpage
\mbox{}
\clearpage
\mbox{}
\clearpage
\mbox{}
\clearpage
\mbox{}
\clearpage
\mbox{}
\clearpage
\mbox{}
\clearpage
\mbox{}
\clearpage
\mbox{}
\clearpage
\mbox{}
\clearpage
\mbox{}
\clearpage
\mbox{}
\clearpage
\mbox{}
 

\clearpage
\limitpages{4}
%\HDR 


 

\clearpage
\mbox{}
\clearpage
\mbox{}
\clearpage
\mbox{}
\clearpage
\mbox{}
\clearpage
\mbox{}
\clearpage
\mbox{}
\clearpage
\mbox{}
\clearpage
\mbox{}
\clearpage
\mbox{}
\clearpage
\mbox{}
\clearpage
\mbox{}
\clearpage
\mbox{}
\clearpage
\mbox{}
\clearpage
\mbox{}
\clearpage
\mbox{}
\clearpage
\mbox{}
 

\clearpage

\limitpages{35}


% \ganttset{group/.append style={orange},
% milestone/.append style={red},
% progress label node anchor/.append style={text=red}}
% \begin{figure}[!hb]
%   \centering
%   \begin{ganttchart}[%Specs
%     expand chart=.85\textwidth,
%      y unit title=0.5cm,
%      y unit chart=0.7cm,
%      vgrid,hgrid,
%      title height=1,
% %     title/.style={fill=none},
%      title label font=\bfseries\footnotesize,
%      bar/.style={fill=blue},
%      bar height=0.7,
% %   progress label text={},
%      group right shift=0,
%      group top shift=0.7,
%      group height=.3,
%      group peaks width={0.2},
%      inline]{1}{24}
%     %labels
%     \gantttitle{A two-years project}{24}\\  % title 1
%     \gantttitle[]{Year 1}{12}                 % title 2
%     \gantttitle[]{Year 2}{12} \\              
%     \gantttitle{Q1}{3}                      % title 3
%     \gantttitle{Q2}{3}
%     \gantttitle{Q3}{3}
%     \gantttitle{Q4}{3}
%     \gantttitle{Q1}{3}
%     \gantttitle{Q2}{3}
%     \gantttitle{Q3}{3} 
%     \gantttitle{Q4}{3}\\
%     % Setting group if any
%     \ganttgroup[inline=false]{UCM Data Acquisition}{1}{3}\\ 
%     \ganttmilestone[inline=false]{IRB Approval}{2} \\
%     \ganttgroup[inline=false]{SSC Biosample Acquisition}{1}{4}\\ 
%     \ganttbar[progress=50,inline=false]{Biosample sequencing}{4}{18}\\
%  \ganttbar[progress=75,inline=false]{UCM Data Analysis}{3}{12}\\
%     \ganttmilestone[inline=false]{UCM analysis complete}{12} \\
%     \ganttmilestone[inline=false]{Year 1 Report}{12} \\
%     \ganttbar[progress=100,inline=false]{Biosample sequencing}{4}{18}\\
%       \ganttmilestone[inline=false]{Biosample sequencing complete}{18} \\
%    \ganttbar[progress=75,inline=false]{FInal analysis}{12}{24}\\
%      \ganttbar[progress=100,inline=false]{FInal analysis}{12}{24}\\
%     \ganttmilestone[inline=false]{Final Report}{24} \\
%   \end{ganttchart}
%   \captionN{Timeline and milestones}\label{figtimeline}
% \end{figure}



\clearpage

\section*{List of Abbreviations, Acronyms, and Symbols}

\clearpage

\limitpages{8}

\section*{Data Management Plan}
% 2 page limit
\clearpage
\mbox{}
\clearpage
\mbox{}
\clearpage
\mbox{}
\clearpage
\mbox{}
\clearpage
\mbox{}
\clearpage
\mbox{}
\clearpage
\mbox{}
\clearpage
\mbox{}
\clearpage
\mbox{}
\clearpage
\mbox{}
\clearpage
\mbox{}
\clearpage
\mbox{}
\clearpage
\mbox{}
\clearpage
\mbox{}
\clearpage
\mbox{}
\clearpage
\mbox{}
 
\clearpage
\limitpages{23}

\section*{Technical Abstract}
% 1 page limit

\clearpage

\section*{Lay Abstract}
% 1 page limit



\clearpage
\limitpages{13}

\section*{Statement of Work}
% 3 page limit

\clearpage
\mbox{}
\clearpage
\mbox{}
\clearpage
\mbox{}
\clearpage
\mbox{}
\clearpage
\mbox{}
\clearpage
\mbox{}
\clearpage
\mbox{}
\clearpage
\mbox{}
\clearpage
\mbox{}
\clearpage
\mbox{}
\clearpage
\mbox{}
\clearpage
\mbox{}
\clearpage
\mbox{}
\clearpage
\mbox{}
\clearpage
\mbox{}
\clearpage
\mbox{}
 

\clearpage
\limitpages{14}
\section*{Impact Statement}
% 1 page limit
% Explain why the proposed research project is important and relevant to the FY23 PRMRP
% Topic Area addressed. Describe the FY23 PRMRP Strategic Goal that is addressed in the
% proposed research. Outline the potential impact, either short-term or long-term, of the
% proposed research on the field of study and/or patient care. Describe how the research 
% DOD FY23 Peer Reviewed Medical Discovery Award 26
% has the potential to generate preliminary data that can be used as a foundation for future
% research projects. 
\clearpage

\limitpages{15}
\section*{Relevance to Military Health Statement}
% 1 page limit

\clearpage
\limitpages{100}
\section*{Facilities, Existing Equipment, and Other Resources}

\textbf{The University of Chicago} is a private non-profit institution located on the ethnically-diverse South Side of Chicago that has been a center of advanced learning and research since its inception in 1892. The University of Chicago is comprised of four graduate Divisions (Biological Sciences, Physical Sciences, Social Sciences, and Humanities), six professional schools (Chicago Booth School of Business, Divinity School, Harris School of Public Policy Studies, Law School, Pritzker School of Medicine, and School of Social Service Administration), the Graham School of General Studies, and the undergraduate College. The University has a unique history of organizing around research questions that cross disciplines rather than operating within  disciplinary boundaries. The extent to which this strategy reflects University of Chicago is illustrated by its numerous interdisciplinary Committees, Centers and Institutes (described below). The University of Chicago maintains its commitment to scholarship, teaching, and research through its more than 2100 faculty members and a student population of approximately 15,600 with nearly 2/3 engaged in advanced research and professional study. Through the years, 86 Nobel Laureates (8 are current faculty),  44 members of the National Academy of Sciences, 169 members of the American Academy of Arts and Sciences, and 14 recipients of the National Medal of Science have been associated with the University as students, teachers or research investigators. The University of Chicago is ranked among the world’s top universities by a number of criteria, including the amount of federal research funding received (despite a size much smaller than many of its academic peers). This spirit of discovery, innovation and public service provides a robust foundation for success.

\textbf{South Shore Senior Health Center (SSSC):} The SSSC is a 6800 sq. ft. university-owned geriatric facility ~5 miles south of DCAM, with doorstep free parking in the heart of Chicago’s South Side, specifically in the South Shore district. There are 13 patient exam rooms equipped with an examination bed, ophthalmoscope and otoscope, in addition to an on-site phlebotomist and capability for ECG. There is a large conference room for team meetings and support groups. The Memory Center team meets on Mondays at this site.

\textbf{Center for Care and Discovery:} Completed in 2013, the CCD is a ten-story adult hospital focused on cancer, advanced surgery, high  tech  imaging,  the  neurosciences  and  gastrointestinal  procedures. The building is 1,200,000 GSF with floor plates of 102,000 GSF. The CCD includes 240 private patient rooms, 28 operating rooms (21 initially); an imaging department with 3 CT’s, 2 MRI’s, 1 fluoroscopy room, 2 general radiology rooms, 7 interventional radiology rooms; and a gastroenterology procedures suite with 11 GI procedure rooms, 2 fluoroscopy rooms and 2 bronchoscopy rooms.  The CCD includes an inpatient kitchen, cafeteria, 7th floor sky lobby meeting rooms, ground floor retail space and clinical support services. Two floors are “shelled space” for future expansion of services. The building is centrally located between existing clinical facilities (DCAM, Comer) and new research facilities (KCBD, GCIS, Knapp). The facility is connected to both DCAM and Comer via above and below ground connections. There are procedure rooms, 2 fluoroscopy rooms and 2 bronchoscopy rooms. Neurology and neurosurgery inpatients as well as a Neuro- ICU are contiguously located on one floor of the CCD. 

% \textbf{Department of Neurology:} The Department of Neurology, chaired by Dr. Shyam Prabhakaran, is a nationally recognized leader in neuroscience research. The department consists of 28 primary faculty and 21staff members. The department has multiple funded investigators and over 3 million dollars in extramural research. Along with the university and the BSD, the Department of Neurology places primary emphasis on the training of the next generation of academic leaders in neurology with a special focus on the development of clinician scientists. There has been expansion of the residency from 18 to 21 (7 per year). The department with UCM has launched the Clinical Neurosciences Service Line in 2019, directed by Dr. Prabhakaran, to integrate clinical care in neurology and neurosurgery across the spectrum of neurologic disease including community health and prevention, hospital-based acute care, and post-acute rehabilitation and re-integration in the community. The Service Line partners with the hospital and university to bring together resources including bioinformatics, quality improvement, clinical trials and research, and tele-technologies to enhance patient experience and outcomes.

% \textbf{Office Space:} The Neurology offices are housed in the UCM campus and medical center. It has 3 available conference rooms, teleconference capabilities, video conferencing capabilities both informally via internet-available software, as well as formally with WebEx. Dr. Mastrianni and the Executive Administrator will have private offices, as will the Outreach Core leader, and nurse. The coordinator and other staff members have dedicated workspaces in the Neurology office suite. All offices include: phone, pager, computer, scanner, fax, high-speed internet access and administrative support. Standard software available includes: SPSS, EndNote, Microsoft Access, and Adobe Acrobat Professional. The PD’s 300 sq foot office is equipped with a locked door and ample file storage. The research staff members are also equipped with locking file storage. In addition, the suite is protected by key card restricted access.

% \textbf{Memory Center:} Roughly 650 new patients and 1500 return patients are evaluated each year at the Memory Center. This is expected to rise, as an additional Behavioral Neurologist is being recruited as of this writing. Recruitment of most subjects will come from the Memory Center clinics. Two clinic sites are utilized. 1) the Duchossois Center for Advanced Medicine (DCAM) on the University of Chicago Medicine campus, and; 2) the South Shore Senior Health Center (SSSC), a 6800 sq. ft. university-owned geriatric facility approximately 5 miles south of DCAM, with convenient doorstep parking in the heart of Chicago’s South Side. Within the SSSC, in addition to referrals from the general geriatricians, there are specialty clinics that act as a referral to the Memory Center, including the Successful Aging and Frailty Evaluation (SAFE). The catchment area for the SSSC includes a large segment of south Chicago and northwestern Indiana, which currently has limited access to geriatrics and dementia specialists. Roughly 80\% of patients with cognitive problems at the SSSC are African American. 
\section*{Computational Facilities}


The principal investigators have access to extensive computational facilities available at the University of Chicago to carry out the tasks described.


\textbf{Access to Clinical Data for AI-enabled Analytics:} The ZeD lab (overseen by Professor Chattopadhyay) is housed within the Department of Medicine at the University of Chicago, and has access to the full range of high end computing resources offered by the University of Chicago. In addition, Prof. Chattppadyay's laboratory has access to the HIPAA compliant clinical data warehouse maintained by the Biological Sciences Division as detailed below:

\textbf{The Clinical Research Data Warehouse:} (CRDW) within the Biomedical Sciences Division of the University of Chicago is one of the deepest, richest, and most research-ready data repositories of its kind. Containing more than a decade of University of Chicago medical data, it seamlessly brings together multiple internal and external data sources to provide researchers with access to more than 12 million encounters for 2.3 million patients. The associated diagnoses, labs, medications, and procedures number in the tens of millions each. The CRDW is run on IBM Netezza Pure Data System for Analytics servers, a patented Asymmetric Massively Parallel Processing architecture designed to deliver exceptional query performance and modular scalability on highly complex mixed workloads.

In order to meet the acute need for data related to COVID-19, the CRDW team has constructed three data marts (de-identified, limited, and identified) to provide the most commonly requested data elements for this patient population. The initial instance of the COVID-19 data mart includes de-identified structured data on patient demographics, encounters, diagnoses, labs, medications, flow sheets, and procedures. Additional data will be added based on resource availability and urgency.

\textbf{Cohort Discovery Tool:} The purpose of this tool (SEE Cohorts) is to provide a secure web-based tool for the initial exploration of de-identified data. It allows researchers to search available data, build a cohort of patients, and view actual de-identified data within the interface. The data in SEE Cohorts is refreshed weekly.

\textbf{Research Computing Center:} The University of Chicago Research Computing Center (RCC) provides high-end research computing resources to researchers at the University of Chicago, which include high-performance computing and visualization resources; high-capacity storage and backup; software; high-speed networking; and hosted data sets. Resources are centrally managed by RCC staff who ensure the accessibility, reliability, and security of the compute and storage systems. A high-throughput network connects the Midway Compute Cluster to the UChicago campus network and the public internet through a number of high-bandwidth uplinks. To support data-driven research RCC hosts a number of large datasets to be accessed within the RCC compute environment.

RCC maintains three pools of servers for distributed high-performance computing. Ideal for tightly coupled parallel calculations, tightly-coupled nodes are linked by a fully non-blocking FDR-10 Infiniband interconnect. Loosely-coupled nodes are similar to the tightly-coupled nodes, but are connected with GigE rather than Infiniband and are best suited for high-throughput jobs. Finally,  shared memory nodes contain much larger main memories (up to 1 TB) and are ideal for memory-bound computations. The types of CPU architectures RCC maintains are tabulated in Table~\ref{tabcap}.

\begin{table}
  \centering
  
\captionN{University of Chicago Research Computing Center Capabilities Summary}\label{tabcap}
\setlength{\arrayrulewidth}{1pt}
\sffamily\fontsize{9}{9}\selectfont
\begin{tabular}{||C{.8in}|c|R{1.85in}|R{.5in}|c||}\hline
\rowcolor{SeaGreen1}Cluster	&Partition&	Compute cores (CPUs)&	Memory	&Other configuration details\\\hline
midway1	&westmere&	12 x Intel X5675 3.07 GHz&	24 GB	& \\\hline
 	&sandyb	&16 x Intel E5-2670 2.6GHz&	32 GB	& \\\hline
 &	bigmem&	16 x Intel E5-2670 2.6GHz&	256 GB	& \\\hline
 	& &	32 x Intel E7-8837 2.67GHz&	1 TB&	 \\\hline
 	&gpu&	16 x Intel E5-2670 2.6GHz&	32 GB&	2 x Nvidia M2090 or K20 GPU\\\hline
 &	 &	20 x Intel E5-2680v2 2.8GHz&	64 GB&	2 x Nvidia K40 GPU\\\hline
 &	mic&	16 x Intel E5-2670 2.6GHz&	32 GB&	2 x Intel Xeon Phi 5100 coprocessor\\\hline
 &	amd&	64 x AMD Opteron 6386 SE&	256 GB&	 \\\hline
 &	ivyb&	20 x Intel E5-2680v2 2.8GHz&	64 GB&	 \\\hline
midway2&	broadwl&	28 x Intel E5-2680v4 2.4GHz&	64 GB&	 \\\hline
 &	bigmem2&	28 x Intel E5-2680v4 @ 2.4 GHz&	512 GB	 &\\\hline
 	&gpu2&	28 x Intel E5-2680v4 @ 2.4 GHz&	64 GB&	4 x Nvidia K80 GPU\\\hline
\end{tabular}
\end{table}


RCC also maintains a number of specialty nodes: 
\begin{itemize}
\item \textit{\color{gray} Large shared memory nodes} - up to 1 TB of memory per node with either 16 or 32 Intel CPU cores. Midway is always expanding, but at time of writing RCC contains a total of 13,500 cores across 792 nodes, and 1.5 PB of storage.
\item  \textit{\color{gray}  Hadoop:} Originally developed at Google, Hadoop is a framework for large-scale data processing.   \item  \textit{\color{gray}  GPU Computing:} Scientific computing on graphics cards can unlock even greater amounts of parallelism from code. RCC GPU nodes each include two Nvidia Tesla-class accelerator cards and are integrated in the Infiniband network. RCC currently provides access to Fermi-generation M2090 GPU devices and Kepler-generation K20 and K40 devices.  \item  \textit{\color{gray}  Xeon Phi:} The Many Integrated-Core architecture (MIC) is Intel's newest approach to manycore computing. Researchers can experiment with these accelerators by using  MIC nodes, each of which have two Xeon Phi cards, and are integrated into the Infiniband network.
\end{itemize}

\textbf{Persistent and High-Capacity Storage.} Storage is accessible from all compute nodes on Midway1 and Midway2 as well as outside of the RCC compute environment through various mechanisms, such as mounting directories as network drives on your personal computer or accessing data as a Globus Online endpoint (at the time of this writing, Globus Online is supported on Midway1). RCC takes snapshots of all home directories (users' private storage space) at regular intervals so that if any data is lost or corrupted, it can easily be recovered. RCC maintains GPFS Filesystem Snapshots for quick and easy data recovery.  In the event of catastrophic storage failure, archival tape backups can be used to recover data from persistent storage locations on Midway. Automated snapshots of the home and project directories are available in case of accidental file deletion or other problems. Currently snapshots are available for these time periods: 1) 7 daily snapshots, 2) 4 weekly snapshots.

\textbf{Tape Backups.} Backups are performed on a nightly basis to a tape machine located in a different data center than the main storage system. These backups are meant to safeguard against events such as hardware failure or disasters that could result in the complete loss of RCC’s primary data center.


\textbf{Data Sharing.} All data in RCC's storage environment is accessible through a wide range of tools and protocols. Because RCC provides centralized infrastructure, all resources are accessible by multiple users simultaneously, which makes RCC’s storage system ideal for sharing data among your research group members. Additionally, data access and restriction levels can be put in place on an extremely granular level.

\textbf{Data Security \& Management.} The HIPAA compliant security of the Research Computing Center’s storage infrastructure, protected by two-factor authentication,  gives users peace of mind that their data is stored, managed, and protected by HPC professionals. Midway's file management system allows researchers to control access to their data. RCC has the ability to develop data access portals for different labs and groups.



\clearpage

\section*{Publications and/or Patents}

\clearpage

\section*{Letters of Organizational Support}

\clearpage

\section*{Letters of Collaboration}
\clearpage

\section*{Intellectual Property}

% Intellectual and Material Property Plan (if applicable): Provide a plan for
% resolving intellectual and material property issues among participating
% organizations.
%  Commercialization Strategy (if applicable): Describe the commercialization
% plan. The plan should include intellectual property, market size, financial
% analysis, strengths and weaknesses, barriers to the market, competitors, and
% management team. Discuss the significance of this development effort, when it
% can be anticipated, and the potential commercial use for the technology being
% developed.



\clearpage

\section*{Data and Research Resources Sharing Plan}

\clearpage

%\clearpage
%\section*{Public Health Service (PHS) Inclusion Enrollment Report}




%\clearpage
%\section*{Representations}




\clearpage

%\bibliographystyle{plainnat}
%\bibliography{asdgenenew}
\bibliographystyle{naturemag}
\bibliography{allbib}


\end{document}

