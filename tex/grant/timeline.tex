\documentclass[onecolumn, compsoc,11pt]{IEEEtran}
\usepackage{enumitem}
\input{preamble.tex}
%\usepackage{jneurosci}
\usepackage{pgfgantt}
% \usepackage{cite}
\usepackage{textcomp}
\usepackage{colortbl}\usepackage{subfigure}
\usepackage{array}
\usepackage{courier}
\usepackage{setspace} 
\usepackage{wrapfig} 
\usepackage{calligra}
\usepackage{ulem}
\usepackage{multirow}

\newcommand*{\doi}[1]{\href{http://dx.doi.org/#1}{doi: #1}}
\renewcommand{\IEEEbibitemsep}{20pt plus 2pt}
\makeatletter
\IEEEtriggercmd{\reset@font\normalfont\fontsize{11}{14}\selectfont}
\makeatother
\IEEEtriggeratref{1}
\newlength{\bibitemsep}\setlength{\bibitemsep}{.2\baselineskip plus .05\baselineskip minus .05\baselineskip}
\newlength{\bibparskip}\setlength{\bibparskip}{0pt}
\let\oldthebibliography\thebibliography
\renewcommand\thebibliography[1]{%
  \oldthebibliography{#1}%
  \setlength{\parskip}{\bibitemsep}%
  \setlength{\itemsep}{\bibparskip}%
}
\setlength{\bibitemsep}{.3\baselineskip plus .05\baselineskip minus .05\baselineskip}

\usetikzlibrary{chains,backgrounds}
\usetikzlibrary{intersections}
% \usepackage[super]{cite} 
% \makeatletter \renewcommand{\@citess}[1]{\raisebox{1pt}{\textsuperscript{[#1]}}} \makeatother
\usepackage{xstring}
\usepackage{wasysym}
\usepackage[misc]{ifsym}
\renewcommand{\thesectiondis}{\arabic{section}.}
\renewcommand{\thesubsectiondis}{\Alph{subsection}.}

\makeatletter
\renewcommand\section{\@startsection {section}{1}{\z@}%
  {-1pt \@plus -30ex \@minus 20ex}%
  {.1pt}%
  {\large\bfseries\scshape}}
\renewcommand\subsection{\@startsection {subsection}{2}{\z@}%
  {0ex \@plus -1.75ex \@minus -1.2ex}%
  {0ex \@plus.0ex}%
  {\fontsize{11}{11}\selectfont\bfseries\sffamily\color{black}}}
\renewcommand\subsubsection{\@startsection {section}{1}{\z@}%
  {-1.5ex \@plus -.5ex \@minus -.2ex}%
  {0.0ex \@plus.5ex}%
  {\fontsize{9}{9}\selectfont\bfseries\sffamily\color{Red4}}}
\renewcommand\paragraph{\@startsection {section}{1}{\z@}%
  {-.1ex \@plus -.5ex \@minus -.2ex}%
  {0.0ex \@plus.5ex}%
  {\fontsize{11}{10}\selectfont\bfseries\itshape\sffamily\color{black}}}
\makeatother


\makeatletter
\pgfdeclareradialshading[tikz@ball]{ball}{\pgfqpoint{-10bp}{10bp}}{%
  color(0bp)=(tikz@ball!30!white);
  color(9bp)=(tikz@ball!75!white);
  color(18bp)=(tikz@ball!90!black);
  color(25bp)=(tikz@ball!70!black);
  color(50bp)=(black)}
\makeatother
\newcommand{\tball}{${\color{CadetBlue3}\Large\boldsymbol{\blacksquare}}$}
\renewcommand{\baselinestretch}{.94}
\newcommand{\VSP}{\vspace{-2pt}}
\renewcommand{\captionN}[1]{\caption{\color{black} \sffamily \fontsize{9}{10}\selectfont #1  }}
\tikzexternaldisable 
\parskip=4pt
\parindent=0pt
\newcommand{\Mark}[1]{\textsuperscript{#1}}
\lhead{}
\pagestyle{fancy}
\def\COLA{black}
% ###################################
\cfoot{\bf\sffamily \scriptsize \color{Maroon!50} I. Chattopadhyay, Department of Medicine, University of Chicago}
\cfoot{}
\rhead{}
% \rhead{\bf\sffamily \scriptsize \color{DodgerBlue4!50} DARPA Young Faculty Award 2017}
% \rhead{\scriptsize\bf\sffamily \href{zed.UChicago.edu}{zed.UChicago.edu}}
% \rfoot{\scriptsize\bf\sffamily\thepage}
\newcommand{\partxt}{\bf\sffamily\itshape}
% ############################################################
\newif\iftikzX
\tikzXtrue
\tikzXfalse

\newcommand\guline{\bgroup\markoverwith
  {\textcolor{black!30}{\rule[-0.45ex]{2pt}{0.4pt}}}\ULon}
\newcommand\hilit[1]{\textcolor{Red1}{#1}}
\newcommand\hilitx[1]{\guline{#1}}
% ############################################################
\addtolength{\voffset}{.1in}
\addtolength{\textwidth}{-.085in}
\addtolength{\hoffset}{.0425in}
\def\PROG{Mallinckrodt\xspace}
\def\ZERO{ACoR\xspace}
\def\COLWA{\XCOLA!40}
\def\COLWB{\XCOLD!20}
\def\COLWC{\XCOLA!40}
\def\COLWD{\XCOLD!20}
\def\COLWE{\XCOLA!40}
\def\COLWF{\XCOLD!20}
% ############################################################
\def\treatment{positive\xspace}
\def\TITLE{A pharmacogenomic hypothesis for reducing the population risk of autism via potential off-label non-contra-indicated use of a common SSRI agent in pregnant women}
\def\TITLE{A pharmacogenomic hypothesis for reducing   risk of autism}
\def\TOTALCOSTSAMPLES{$\$1875 \times 7 = \$13,125$}
\def\PINAME{Ishanu Chattopadhyay}
\def\PIINST{University of Chicago}
\def\PIEMAIL{\url{ishanu@uchicago.edu}}

\def\SIMONDATA{SSC proband \& matched designated sibling ($672$ samples)}

\def\acor{ACoR\xspace}

\newcommand{\HDR}{
\begin{tabular}{|L{.3\textwidth} | L{.32\textwidth} | L{.3\textwidth} | }\hline
  Principal investigator Name: \bf \PINAME & Principal Investigator Institution: \bf \PIINST & Principal Investigator  email: \PIEMAIL \\\hline
  \multicolumn{2}{|C{.62\textwidth}|}{\hspace{-15pt} \mnp{.63\textwidth}{\vskip .45em Project Title\\ \bf \TITLE \\\vspace{-8pt} }} &  {\mnp{.3\textwidth}{\vskip .3em Project Type: \\ \bf Pilot}} \\\hline
  \multicolumn{3}{|C{.92\textwidth}|}{ \hspace{-40pt} \mnp{.93\textwidth}{\vskip .6em Data requested from Simons Collection:  \SIMONDATA \\ \vspace{-8pt}}} \\\hline
  \multicolumn{2}{|C{.62\textwidth}|}{\hspace{-15pt} \mnp{.63\textwidth}{Total Estimated Cost for Samples (Price List)}}& \TOTALCOSTSAMPLES\\\hline
\end{tabular}
\vskip .5em
}


\begin{document} 



% \subsection*{Timeline and milestones}
% The project will be carried out at the University of Chicago over 2 years at a total cost of 300K, and progress will be tracked by 5 milestones% : 1) IRB approval, 2) complete UCM data analysis, 3) Year 1 report, 4) Sequencing analysis complete, and 5) Final  report
% (See Gantt chart in Fig.~\ref{figtimeline}).

% Please fill in all header fields above. The instructions below may be deleted.
% This Proposal Narrative should not exceed three (3) pages of single-spaced, size 11 text, 0.5 margins. Figures, figure
% legends and references should follow the narrative text, and will NOT count towards the page limit. References should
% be in Journal of Neuroscience format, including full author list, title, and a link to PubMed.
% The Proposal Narrative should be considered an expanded version of the Specific Aims Page and used to provide more
% detail on the following:
%  Relevant scientific background and relevance to autism
%  Preliminary results when applicable
%  Specific aims
%  Experimental design
%  Pitfalls and alternative strategies
%  Future directions and implications for autism diagnosis, understanding, or treatment
%  Timeline and milestones
% Applicants should not devote too much space to the background section; there is no need to provide a scholarly review
% of autism.
% SFARI considers the following information crucial for the evaluation of a project and encourages including the following
% details (where relevant):
%  iPS lines and controls to be used and their availability. For reference, please see SFARI’s information sheet on
% Experimental Design Considerations for IPSC models (available in SAM)
%  Animal model strain/lines and their availability
%  Patient cohorts used (including source of participants, sample size, and availability of genetic data)
%  Datasets or biospecimen collections to be used (including SFARI resources) and their availability
%  A brief statement of statistical power. For reference, please see SFARI’s information sheet on Methodological
% and Statistical Considerations (available in SAM)

 



\ganttset{group/.append style={orange},
milestone/.append style={red},
progress label node anchor/.append style={text=red}}
\begin{figure}[!hb]
 %\tikzexternalenable
  %\tikzsetnextfilename{timeline}

  \centering
  \begin{ganttchart}[%Specs
    expand chart=.85\textwidth,
     y unit title=0.5cm,
     y unit chart=0.7cm,
     vgrid,hgrid,
     title height=1,
%     title/.style={fill=none},
     title label font=\bfseries\footnotesize,
     bar/.style={fill=blue},
     bar height=0.7,
%   progress label text={},
     group right shift=0,
     group top shift=0.7,
     group height=.3,
     group peaks width={0.2},
     inline]{1}{24}
    %labels
    \gantttitle{Project Duration: 2 years}{24}\\  % title 1
    \gantttitle[]{Year 1}{12}                 % title 2
    \gantttitle[]{Year 2}{12} \\              
    \gantttitle{Q1}{3}                      % title 3
    \gantttitle{Q2}{3}
    \gantttitle{Q3}{3}
    \gantttitle{Q4}{3}
    \gantttitle{Q1}{3}
    \gantttitle{Q2}{3}
    \gantttitle{Q3}{3} 
    \gantttitle{Q4}{3}\\
    % Setting group if any
    %\ganttgroup[inline=false]{UCM Data}{1}{3}\\ 
    \ganttmilestone[inline=false]{IRB Approval}{2} \\
    \ganttgroup[inline=false]{Patient recruitment}{3}{20}\\ 
    \ganttbar[progress=50,inline=false]{ACoR Analysis}{4}{20}\\
    \ganttbar[progress=40,inline=false]{ADOS-2 scheduling}{6}{22}\\
 %\ganttbar[progress=75,inline=false]{UCM Data Analysis}{3}{12}\\
   % \ganttmilestone[inline=false]{UCM analysis complete}{12} \\
    \ganttmilestone[inline=false]{Interim analysis}{12} \\
    \ganttmilestone[inline=false]{Year 1 Report}{12} \\
    \ganttbar[progress=100,inline=false]{ACoR Analysis}{4}{20}\\
    \ganttbar[progress=100,inline=false]{ADOS-2 scheduling}{6}{22}\\

    \ganttmilestone[inline=false]{Final Report}{24} \\
  \end{ganttchart}
 % \captionN{Timeline and milestones}\label{figtimeline}
\end{figure}





\end{document}

